\everymath{\displaystyle}
%\documentclass[pdftex,a4paper]{article}
\documentclass[a4paper]{article}
%%classes: article, report, book, proc, amsproc

%%%%%%%%%%%%%%%%%%%%%%%%
%% Misc
% para acertar os acentos
\usepackage[brazilian]{babel} 
%\usepackage[portuguese]{babel} 
% \usepackage[english]{babel}
% \usepackage[T1]{fontenc}
% \usepackage[latin1]{inputenc}
\usepackage[utf8]{inputenc}
\usepackage{indentfirst}
\usepackage{fullpage}
% \usepackage{graphicx} %See PDF section
\usepackage{multicol}
\setlength{\columnseprule}{0.5pt}
\setlength{\columnsep}{20pt}
%%%%%%%%%%%%%%%%%%%%%%%%
%%%%%%%%%%%%%%%%%%%%%%%%
%% PDF support

\usepackage[pdftex]{color,graphicx}
% %% Hyper-refs
\usepackage[pdftex]{hyperref} % for printing
% \usepackage[pdftex,bookmarks,colorlinks]{hyperref} % for screen

%% \newif\ifPDF
%% \ifx\pdfoutput\undefined\PDFfalse
%% \else\ifnum\pdfoutput > 0\PDFtrue
%%      \else\PDFfalse
%%      \fi
%% \fi

%% \ifPDF
%%   \usepackage[T1]{fontenc}
%%   \usepackage{aeguill}
%%   \usepackage[pdftex]{graphicx,color}
%%   \usepackage[pdftex]{hyperref}
%% \else
%%   \usepackage[T1]{fontenc}
%%   \usepackage[dvips]{graphicx}
%%   \usepackage[dvips]{hyperref}
%% \fi

%%%%%%%%%%%%%%%%%%%%%%%%


%%%%%%%%%%%%%%%%%%%%%%%%
%% Math
\usepackage{amsmath,amsfonts,amssymb}
% para usar R de Real do jeito que o povo gosta
\usepackage{amsfonts} % \mathbb
% para usar as letras frescas como L de Espaco das Transf Lineares
% \usepackage{mathrsfs} % \mathscr

% Oferecer seno e tangente em pt, com os comandos usuais.
\providecommand{\sin}{} \renewcommand{\sin}{\hspace{2pt}\mathrm{sen}}
\providecommand{\tan}{} \renewcommand{\tan}{\hspace{2pt}\mathrm{tg}}

% dt of integrals = \ud t
\newcommand{\ud}{\mathrm{\ d}}
%%%%%%%%%%%%%%%%%%%%%%%%



\begin{document}

%%%%%%%%%%%%%%%%%%%%%%%%
%% Título e cabeçalho
%\noindent\parbox[c]{.15\textwidth}{\includegraphics[width=.15\textwidth]{logo}}\hfill
\parbox[c]{.825\textwidth}{\raggedright%
  \sffamily {\LARGE

Bioestatística: Roteiro para apresentação dos seminários

\par\bigskip}
{Prof: Felipe Figueiredo\par}
{\url{http://sites.google.com/site/proffelipefigueiredo}\par}
}

Versão: \verb|20150518|

%%%%%%%%%%%%%%%%%%%%%%%%


%%%%%%%%%%%%%%%%%%%%%%%%

\section{Roteiro para apresentação dos seminários}

Como forma de avaliação da disciplina de Bioestatística, os alunos
devem interpretar criticamente a análise estatística empregada em um
artigo de pesquisa a ser selecionado de uma lista. Os alunos podem
sugerir artigos que estejam dentro do escopo da ementa da disciplina,
ou selcionar um artigo da lista providenciada pelo Docente. A lista de
artigos sugeridos, suas referências e links para o PDF serão
disponibilizadas em
\url{https://sites.google.com/site/proffelipefigueiredo/into/bioestatistica}.

Os alunos deverão preparar uma apresentação em dupla do estudo
selecionado. O objetivo da avaliação é que os alunos façam uma síntese
crítica da análise estatística utilizada neste estudo à luz dos
conteúdos vistos ao longo da disciplina \footnote{Declaração: O
  roteiro acima foi desenvolvido baseado na atividade de fim de curso
  da prof. Emília Sakurai (RTE-02/2003 - UFMG).}.

Cada dupla deve observar as questões a seguir na elaboração de suas
apresentações.

\begin{itemize}
\item Os objetivos do artigo selecionado. Estão declarados de forma
  clara no estudo?
\item Como você classificaria o tipo do estudo (observacional,
  experimental, etc)?
\item Os elementos de análise estatística descritiva utilizados
  (média, desvio-padrão, etc.)
\item Qual é a variável resposta (dependente) do estudo? Como você as
  classificaria? Qual a unidade desta variável?
\item Qual ou quais são as variáveis explicativas (independentes) do
  estudo. Como você as classificaria? Quais as unidades destas
  variáveis?
\item Os elementos de análise inferencial, caso seja aplicável
  (intervalos de confiança, etc). Os testes estatísticos utilizados
  adequados para o tamanho da amostra e para o tipo da variável?
\item Foi feito um teste de hipóteses no artigo? As hipóteses estão
  descritas de forma clara e precisa?
\item Os resultados são estatisticamente significativos? Os autores
  apresentam o p-valor, ou apenas identificam a significância? O nível
  de significância está explícito no artigo?
\end{itemize}

A dupla deve iniciar com a referência completa do artigo, e discutir a
pertinência do título: este é informativo? Ele resume bem os objetivos
e resultados do estudo?

Cada dupla será avaliada durante 30 minutos, sendo 20 minutos para
apresentação do artigo e 10 minutos para discussão e arguição.

\end{document}
