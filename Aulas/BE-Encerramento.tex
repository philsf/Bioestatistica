\everymath{\displaystyle}
\documentclass{beamer}
% \documentclass[handout]{beamer}

%\usepackage[pdftex]{color,graphicx}
\usepackage{amsmath,amssymb,amsfonts}

\mode<presentation>
{
  % \usetheme{Darmstadt}
  % \usetheme[hideothersubsections]{Hannover}
  % \usetheme[hideothersubsections]{Goettingen}
  \usetheme[hideothersubsections, right]{Berkeley}

  \usecolortheme{seahorse}
  % \usecolortheme{dolphin}
  \usecolortheme{rose}
  % \usecolortheme{orchid}

  \useinnertheme[shadow]{rounded}

  % \setbeamercovered{transparent}
  \setbeamercovered{invisible}
  % or whatever (possibly just delete it)
}

\mode<handout>{
  \setbeamercolor{background canvas}{bg=black!5}
  \usepackage{pgfpages}
  \pgfpagesuselayout{4 on 1}[a4paper,border shrink=5mm, landscape]
}

\usepackage[brazilian]{babel}
% or whatever

% \usepackage[latin1]{inputenc}
\usepackage[utf8]{inputenc}
% or whatever

\usepackage{times}
%\usepackage[T1]{fontenc}
% Or whatever. Note that the encoding and the font should match. If T1
% does not look nice, try deleting the line with the fontenc.


\title%[] % (optional, use only with long paper titles)
{}

\subtitle
{} % (optional)

\author%[] % (optional, use only with lots of authors)
{Felipe Figueiredo}% \and S.~Another\inst{2}}
% - Use the \inst{?} command only if the authors have different
%   affiliation.

\institute[] % (optional, but mostly needed)
{
}
  % \inst{1}%
  % Department of Computer Science\\
  % University of Somewhere
  % \and
  % \inst{2}%
  % Department of Theoretical Philosophy\\
  % University of Elsewhere}
% - Use the \inst command only if there are several affiliations.
% - Keep it simple, no one is interested in your street address.

\date%[] % (optional)
{}

% \subject{Talks}
% This is only inserted into the PDF information catalog. Can be left
% out. 



% If you have a file called "university-logo-filename.xxx", where xxx
% is a graphic format that can be processed by latex or pdflatex,
% resp., then you can add a logo as follows:

\pgfdeclareimage[height=1.6cm]{university-logo}{../logo}
\logo{\pgfuseimage{university-logo}}



% Delete this, if you do not want the table of contents to pop up at
% the beginning of each subsection:
\AtBeginSubsection[]
%\AtBeginSection[]
{
  \begin{frame}<beamer>{Sumário}
    \tableofcontents[currentsection,currentsubsection]
  \end{frame}
}


% If you wish to uncover everything in a step-wise fashion, uncomment
% the following command: 

% \beamerdefaultoverlayspecification{<+->}

\usepackage[normalem]{ulem}

\begin{document}

\begin{frame}
  \titlepage
\end{frame}

\begin{frame}{Sumário}
  \tableofcontents
  % You might wish to add the option [pausesections]
\end{frame}


%% Template
% \section{}

% \subsection{}

% \begin{frame}{}
%   \begin{itemize}
%   \item 
%   \end{itemize}
% \end{frame}

% \begin{frame}
%   \begin{columns}
%     \begin{column}{5cm}
%     \end{column}
%     \begin{column}{5cm}
%     \end{column}
%   \end{columns}
% \end{frame}

% \begin{frame}{}
%   \includegraphics[height=0.4\textheight]{file1}
%   \includegraphics[height=0.4\textheight]{file2}
%   \includegraphics[height=0.4\textheight]{file3}
%   \begin{figure}
%     \caption{}
%   \end{figure}
% \end{frame}

% \begin{frame}{}
%   \begin{definition}
%   \end{definition}
%   \begin{example}
%   \end{example}
%   \begin{block}{Exercício}
%   \end{block}
% \end{frame}

\section{Aulas}

\subsection{As aulas}

\begin{frame}{\scriptsize As aulas}
  \begin{enumerate}
    \tiny
  \item Introdução
  \end{enumerate}

  % {\tiny Análise de Dados}
  % \bigskip
  \begin{block}{\scriptsize Módulo 1}
    \begin{enumerate}
      \setcounter{enumi}{1}
      \tiny
    \item Intervalos de Confiança de proporções

      % {\tiny Incertezas de dados categóricos}
    \item Variabilidade

      % {\tiny Incertezas de dados numéricos}
    \item A distribuição Normal

      % {\tiny Distribuição Normal, e IC da média}
    \end{enumerate}
  \end{block}
  \begin{block}{\scriptsize Módulo 2}
    \begin{enumerate}
      \setcounter{enumi}{4}
      \tiny
    \item Comparando médias de 2 grupos

      % {\tiny Intervalos de Confiança da diferença entre as médias}
    \item Comparando ICs de proporções

      % {\tiny A Razão de Chances e o Risco Relativo}
    \item Significância e Poder

      % {\tiny Testes de Hipóteses, cálculo amostral e p-valor}
    \item Comparação de dois grupos (quantitativo)

      % {\tiny Testes paramétricos para médias}
    \item Comparação de dois grupos (qualitativo)

      % {\tiny Testes para proporções}
    \end{enumerate}
  \end{block}
  \begin{block}{\scriptsize Módulo 3}
    \begin{enumerate}
      \setcounter{enumi}{8}
      \tiny
    \item Correlação Linear

      % {\tiny Associação de duas amostras (quantitativa)}
    \item Regressão Linear Simples

      % {\tiny Modelos com desfecho contínuo}
    \item Tópicos em Regressão Logística

      % {\tiny Modelos com desfecho categórico binário}
    \item Comparações múltiplas e ANOVA

      % {\tiny Teste paramétrico para vários grupos (desfecho quantitativo)}
    \end{enumerate}
  \end{block}
  \begin{enumerate}
    \setcounter{enumi}{13}
    \tiny
  \item Métodos não paramétricos

    % {\tiny Ou: o que fazer caso seus dados não sejam normais?}
  \end{enumerate}
\end{frame}

\subsection{Módulo 1}

\begin{frame}{\scriptsize Módulo 1 -- aulas}
  \begin{enumerate}
    \setcounter{enumi}{1}
  \item Intervalos de Confiança de proporções

    {\tiny Incertezas de dados categóricos}
    \bigskip
  \item Variabilidade

    {\tiny Incertezas de dados numéricos}
    \bigskip
  \item A distribuição Normal

    {\tiny Distribuição Normal, e IC da média}
  \end{enumerate}
\end{frame}

\subsection{Módulo 2}

\begin{frame}{\scriptsize Módulo 2 -- aulas}
  \begin{enumerate}
    \setcounter{enumi}{4}
  \item Comparando médias de 2 grupos

    {\tiny Intervalos de Confiança da diferença entre as médias}
    \bigskip
  \item Comparando ICs de proporções

    {\tiny A Razão de Chances e o Risco Relativo}
    \bigskip
  \item Significância e Poder

    {\tiny Testes de Hipóteses, cálculo amostral e p-valor}
    \bigskip
  \item Comparação de dois grupos (quantitativo)

    {\tiny Testes paramétricos para médias}
    \bigskip
  \item Comparação de dois grupos (qualitativo)

    {\tiny Testes para proporções}

  \end{enumerate}
\end{frame}

\subsection{Módulo 3}

\begin{frame}{\scriptsize Módulo 3 -- aulas}
  \begin{enumerate}
    \setcounter{enumi}{9}
  \item Correlação Linear

    {\tiny Associação de duas amostras (quantitativa)}
    \bigskip
  \item Regressão Linear Simples

    {\tiny Modelos com desfecho contínuo}
    \bigskip
  \item Tópicos em Regressão Logística

    {\tiny Modelos com desfecho categórico binário}
    \bigskip
  \item Comparações múltiplas e ANOVA

    {\tiny Teste paramétrico para vários grupos (desfecho quantitativo)}
  \end{enumerate}
\end{frame}

\end{document}
