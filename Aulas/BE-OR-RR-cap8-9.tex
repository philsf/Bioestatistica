\everymath{\displaystyle}
\documentclass{beamer}
% \documentclass[handout]{beamer}

%\usepackage[pdftex]{color,graphicx}
\usepackage{amsmath,amssymb,amsfonts}

\mode<presentation>
{
  % \usetheme{Darmstadt}
  % \usetheme[hideothersubsections]{Hannover}
  % \usetheme[hideothersubsections]{Goettingen}
  \usetheme[hideothersubsections, right]{Berkeley}

  \usecolortheme{seahorse}
  % \usecolortheme{dolphin}
  \usecolortheme{rose}
  % \usecolortheme{orchid}

  \useinnertheme[shadow]{rounded}

  \setbeamercovered{transparent}
  % \setbeamercovered{invisible}
  % or whatever (possibly just delete it)
}

\mode<handout>{
  \setbeamercolor{background canvas}{bg=black!5}
  \usepackage{pgfpages}
  \pgfpagesuselayout{4 on 1}[a4paper,border shrink=5mm, landscape]
}

\usepackage[brazilian]{babel}
% or whatever

% \usepackage[latin1]{inputenc}
\usepackage[utf8]{inputenc}
% or whatever

\usepackage{times}
%\usepackage[T1]{fontenc}
% Or whatever. Note that the encoding and the font should match. If T1
% does not look nice, try deleting the line with the fontenc.


\title%[] % (optional, use only with long paper titles)
{Comparando ICs de proporções}

\subtitle
{A Razão de Chances e o Risco Relativo} % (optional)

\author%[] % (optional, use only with lots of authors)
{Felipe Figueiredo}% \and S.~Another\inst{2}}
% - Use the \inst{?} command only if the authors have different
%   affiliation.

\institute[INTO] % (optional, but mostly needed)
{Instituto Nacional de Traumatologia e Ortopedia
}
  % \inst{1}%
  % Department of Computer Science\\
  % University of Somewhere
  % \and
  % \inst{2}%
  % Department of Theoretical Philosophy\\
  % University of Elsewhere}
% - Use the \inst command only if there are several affiliations.
% - Keep it simple, no one is interested in your street address.

\date%[] % (optional)
{}

% \subject{Talks}
% This is only inserted into the PDF information catalog. Can be left
% out. 



% If you have a file called "university-logo-filename.xxx", where xxx
% is a graphic format that can be processed by latex or pdflatex,
% resp., then you can add a logo as follows:

\pgfdeclareimage[height=1.6cm]{university-logo}{../logo}
\logo{\pgfuseimage{university-logo}}



% Delete this, if you do not want the table of contents to pop up at
% the beginning of each subsection:
\AtBeginSubsection[]
%\AtBeginSection[]
{
  \begin{frame}<beamer>{Sumário}
    \tableofcontents[currentsection,currentsubsection]
  \end{frame}
}


% If you wish to uncover everything in a step-wise fashion, uncomment
% the following command: 

% \beamerdefaultoverlayspecification{<+->}

\usepackage[normalem]{ulem}

\begin{document}

\begin{frame}
  \titlepage
\end{frame}

\begin{frame}{Sumário}
  \tableofcontents
  % You might wish to add the option [pausesections]
\end{frame}


%% Template
% \section{}

% \subsection{}

% \begin{frame}{}
%   \begin{itemize}
%   \item 
%   \end{itemize}
% \end{frame}

% \begin{frame}
%   \begin{columns}
%     \begin{column}{5cm}
%     \end{column}
%     \begin{column}{5cm}
%     \end{column}
%   \end{columns}
% \end{frame}

% \begin{frame}{}
%   \includegraphics[height=0.4\textheight]{file1}
%   \includegraphics[height=0.4\textheight]{file2}
%   \includegraphics[height=0.4\textheight]{file3}
%   \begin{figure}
%     \caption{}
%   \end{figure}
% \end{frame}

% \begin{frame}{}
%   \begin{definition}
%   \end{definition}
%   \begin{example}
%   \end{example}
%   \begin{block}{Exercício}
%   \end{block}
% \end{frame}

\section{Discussão da aula passada}

\subsection{Discussão da aula passada}

\begin{frame}{Discussão da aula passada}
  \begin{block}{}
    Discussão da leitura obrigatória da aula passada
  \end{block}
\end{frame}

\section{Intro}

\begin{frame}{Exemplos}
  \begin{exampleblock}{8.1 AZT em HIV-positivos assintomáticos}<1>
    Cooper, {\em et al.} estudaram se o AZT diminuía a taxa de progressão da doença em pacientes assintomáticos.
    Participantes adultos, HIV positivos e assintomáticos.
    Participantes distribuídos aleatoriamente em dois grupos que receberam AZT ou um placebo.
    Desfecho de interesse: progressão ou não da síndrome após 3 anos.
  \end{exampleblock}
  \begin{exampleblock}{9.1 Doença da arranhadura de gato}<2>
    Zangwill {\em et al.} estudaram se a doença é mais frequente em donos de gatos que possuem pulgas?
    Cartas a diversos clínicos da região pedindo para relatar casos da doença no último ano.
    Participantes aleatórios (por tel.) que não haviam tido a doença.
    Perguntaram a todos os participantes se seus gatos tinham ou não pulgas.
  \end{exampleblock}
\end{frame}

\begin{frame}[label=exemplos8.1-9.1]{Exemplos}
  \begin{exampleblock}{8.1 AZT em HIV-positivos assintomáticos}
    Cooper, {\em et al.} estudaram se o AZT diminuía a taxa de progressão da doença em pacientes assintomáticos.
    Participantes adultos, HIV positivos e assintomáticos.
    Participantes distribuídos aleatoriamente em dois grupos que receberam AZT ou um placebo.
    Desfecho de interesse: progressão ou não da síndrome após 3 anos.
  \end{exampleblock}
  \begin{exampleblock}{9.1 Doença da arranhadura de gato}
    Zangwill {\em et al.} estudaram se a doença é mais frequente em donos de gatos que possuem pulgas?
    Cartas a diversos clínicos da região pedindo para relatar casos da doença no último ano.
    Participantes aleatórios (por tel.) que não haviam tido a doença.
    Perguntaram a todos os participantes se seus gatos tinham ou não pulgas.
  \end{exampleblock}
\end{frame}

\begin{frame}{Análises de dados categóricos}
  \begin{itemize}
  \item Podemos avaliar as proporções do desfecho de interesse de ambos os estudos
  \item Podemos ainda avaliar estas proporções para cada grupo selecionado
  \item Como comparar estas proporções, em cada caso?
  \item Veremos que são necessárias técnicas distintas!
  \item Mas como decidir qual a técnica apropriada?
  \end{itemize}
\end{frame}

\subsection{Tipos de Estudos}

\begin{frame}{Tipos de estudos}
  \begin{enumerate}
  \item Restrospectivo
  \item Prospectivo
  \item Transversal
  \item Experimental
  \end{enumerate}
  \begin{block}{}
    Cada tipo de estudo consiste em uma série de escolhas metodológicas que restringem como a análise de dados pode ser feita.
  \end{block}
\end{frame}

\begin{frame}{Tipos de estudos}
  \begin{block}{Retrospectivo (ou caso-controle)}
    \begin{itemize}
    \item Parte do desfecho e procuram a causa.
    \item 2 grupos de participantes: 1 com a doença (casos) e 1 sem a doença (controles) - ambos semelhantes nos outros aspectos.
    \item Objetivo: comparar se houve, no passado, exposição diferenciada ao fator de risco considerado.
    \end{itemize}
  \end{block}
  \begin{block}{Prospectivo}<2->
    \begin{itemize}
    \item Parte da exposição e procuram o desfecho.
    \item 2 grupos de participantes: 1 exposto ao fator de risco, e 1 sem exposição.
    \item Objetivo: comparar se haverá diferenciação nas incidências.
    \end{itemize}
  \end{block}
\end{frame}

\begin{frame}{Tipos de estudos}
  \begin{block}{Transversal ({\em cross-sectional})}
    \begin{itemize}
    \item Parte de uma amostra de participantes, sem considerar exposição a fator, ou presença do desfecho.
    \item Divide-se a amostra em 2 grupos baseado na exposição prévia ao fator.
    \item Objetivo: observar a prevalência nos 2 grupos.
    \end{itemize}
  \end{block}
  \begin{block}{Experimental}
    \begin{itemize}
    \item Parte de uma amostra de participantes.
    \item Divide-se a amostra em 2 grupos e administra o tratamento a 1 deles.
    \item Objetivo: observar a incidência nos 2 grupos.
    \end{itemize}
  \end{block}
\end{frame}

\section{Analisando proporções}

\begin{frame}{Tipos de estudos}
  \begin{block}{}
    Classifique os estudos dos exemplos do início da aula

    \begin{itemize}
    \item Observacional ou experimental?
    \item Parte do risco ou do desfecho?
    \item Avalia o passado, presente ou o futuro?
    \item O que mais você pode dizer sobre esses estudos?
    \end{itemize}
  \end{block}
\end{frame}

\againframe{exemplos8.1-9.1}

% \begin{frame}{Características dos estudos dos exemplos}
%   \begin{block}{Sobre o estudo do AZT}
%     \begin{itemize}
%     \item 
%     \end{itemize}
%   \end{block}
%   \begin{block}{Sobre o estudo da Doença da arranhadura de gatos}
%     \begin{itemize}
%     \item 
%     \end{itemize}
%   \end{block}
% \end{frame}

\subsection{Tabelas de Contingência}

\begin{frame}{Dados categóricos}
  \begin{itemize}
  \item Vamos analisar contagens de dados categóricos (nominais)
  \item Para estas variáveis qualitativas não existe ordenação inerente
  \item Observamos apenas as frequências (contagens) destes dados na amostra.
  \end{itemize}
  \begin{exampleblock}{Exemplos}
    doente/sadio, fumante/não fumante, masculino/feminino, olhos
    castanhos/azuis/verdes, etc.
  \end{exampleblock}
\end{frame}

% \begin{frame}{Eventos independentes}
%   Conforme vimos na aula de Probabilidades:
%   \begin{itemize}
%   \item Dois eventos são independentes se a ocorrência do primeiro não
%     afeta a ocorrência do segundo
%   \item Isto significa que a probabilidade da ocorrência do segundo
%     não é condicional em relação ao primeiro
%   \item Em relação aos dados de uma amostra: a frequência observada
%     para cada categoria indica que estas são independentes?
%   \end{itemize}
% \end{frame}

\begin{frame}{Objetivo}
  \begin{exampleblock}{Exemplo 8.1 - apenas contagens}
    Frequências observadas:
    \begin{tabular}{c|c|c}
      & doença progrediu & doença não progrediu\\
      \hline
      AZT & 76 & 399 \\
      \hline
      Placebo & 129 & 332 \\
    \end{tabular}
  \end{exampleblock}
  \begin{block}{Pergunta}
    A partir destes dados é possível determinar se existe alguma relação entre as variáveis?

    Isto é: {\bf as proporções de progressão são diferentes nos dois grupos?}
  \end{block}
\end{frame}

\begin{frame}{Quais são as variáveis?}

  \begin{block}{Duas variáveis categóricas (binárias)}
  \begin{itemize}
  \item {\bf Desfecho} = Progrediu ou Não progrediu
  \item {\bf Grupo de tratamento} = AZT ou Placebo
  \end{itemize}
  \end{block}
  \begin{block}{Esta relação pode ser expressa como}
    \begin{displaymath}
      \text{Desfecho} \sim \text{Grupo de tratamento}
    \end{displaymath}
  \end{block}
\end{frame}

\begin{frame}{Tabela de contingência}
  \begin{exampleblock}{Exemplo 8.1 - apenas contagens}
    Frequências observadas:
    \begin{tabular}{c|c|c}
      & doença progrediu & doença não progrediu\\
      \hline
      AZT & 76 & 399 \\
      \hline
      Placebo & 129 & 332 \\
    \end{tabular}
  \end{exampleblock}
  \begin{block}{Definição}
    Uma \alert{tabela de contingência} mostra as frequências
    observadas para as exposições/tratamentos (linhas) nos desfechos estudados (colunas).
  \end{block}
\end{frame}

\begin{frame}{Tabela de contingência}
  É conveniente considerar a tabela, expandida com os totais de linhas e colunas.
    \begin{exampleblock}{Exemplo 8.1 - totais preenchidos}
    Frequências observadas:
    \begin{tabular}{c|c|c|c}
      & progrediu & não progrediu & total\\
      \hline
      AZT & 76 & 399 & 475\\
      \hline
      Placebo & 129 & 332 & 461\\
      \hline
      total & 205 & 731 & 936\\
    \end{tabular}
  \end{exampleblock}
  \begin{itemize}
  \item Aprendemos a interpretar os ICs de cada proporção
  \item Progressão (AZT): 76/475 = 16\% (IC 95\%: [13\%, 20\%])
  \item Progressão (placebo): 28\% (IC 95\%: [24\%, 32\%])
  \item Como comparar as duas proporções?
  \end{itemize}
\end{frame}

\section{Risco Relativo}

\subsection{Risco Relativo (RR)}

\begin{frame}{Risco Relativo (RR)}
  \begin{block}{Definição}
    Risco relativo (relative risk, RR) é a razão entre os riscos absolutos (proporções), relativos à exposição.
  \end{block}
  \begin{itemize}
  \item RR $> 1 \Rightarrow$ risco aumentado
  \item RR $< 1 \Rightarrow$ risco diminuído
  \item RR $\approx 1 \Rightarrow$ risco semelhante
  \end{itemize}
  \begin{exampleblock}{No exemplo, a progressão foi:}
    \begin{itemize}
    \item 16\% sob AZT e 28\% sob placebo
    \item RR = 0.16/0.28 = 0.57 (interprete)
    \item<2-> IC 95\%: [0.44, 0.75] (interprete)
    \end{itemize}
  \end{exampleblock}
\end{frame}

\begin{frame}{Risco Relativo}
  \begin{exampleblock}{No exemplo, a progressão foi:}
    \begin{itemize}
    \item 16\% sob AZT e 28\% sob placebo
    \item RR = 0.16/0.28 = 0.57 (interprete)
    \item IC 95\%: [0.44, 0.75] (interprete)
    \end{itemize}
  \end{exampleblock}
  \begin{block}{O RR é a comparação entre os dois riscos}
    \small
    \begin{itemize}
    \item Sua interpretação \alert{não é} quanto um risco é maior que o outro.
    \item O correto é concluir qual é o RR \alert{entre} os dois fatores.
    \end{itemize}
  \end{block}
\end{frame}

\begin{frame}{Risco Relativo}
  \begin{exampleblock}{Resultados}
    \begin{itemize}
      \footnotesize
    \item RR = 0.57
    \item IC 95\%: [0.44, 0.75]
    \end{itemize}
  \end{exampleblock}

  \bigskip
  \begin{exampleblock}{Conclusão}
    ... participantes tratados com AZT são 57\% (IC: 44\% a 75\%) tão suscetíveis à progressão... em relação ao placebo
  \end{exampleblock}
\end{frame}

\begin{frame}{Observação}
  \begin{block}{Note que...}
    O RR pode ser calculado como exposto/controle ou controle/exposto.
  \end{block}
  \begin{exampleblock}{AZT}
    \begin{itemize}
%    \item tratam./controle = 0.57
    \item controle/tratam. = 0.28/0.16 = 1.75
    \item {\bf Pacientes que receberam placebo foram 1.75 mais suscetíveis à progressão...}
    \end{itemize}
  \end{exampleblock}
  \begin{block}{Atenção}
    Sempre diga explicitamente qual é a RR que você está considerando!
  \end{block}
\end{frame}

\subsection{Premissas}

\begin{frame}{Premissas da RR}
  Em um estudo prospectivo ou experimental, assumimos:
  \begin{enumerate}
  \item Participantes aleatoriamente amostrados da população (ou pelo menos representativos)
  \item Cada participante é independente dos outros
  \item Única diferença entre os grupos: exposição ao fator de risco
  \end{enumerate}
\end{frame}

\section{Razão de Chances}

\subsection{Razão de Chances (OR)}

\begin{frame}{Chances}
  \begin{block}{Probabilidade}
    Proporção das vezes em que você espera observar um evento, em vários experimentos.
  \end{block}
  \begin{block}{Chance (Odds)}
    Probabilidade de que um evento vai ocorrer, dividida pela probabilidade de que o evento não vai ocorrer.
    \begin{displaymath}
      Odds = \frac{Prob}{1- Prob}
    \end{displaymath}
  \end{block}
  \begin{exampleblock}{AZT, chances de progressão}
    \begin{itemize}
    \item AZT = 0.19
    \item Placebo = 0.39
    \end{itemize}
  \end{exampleblock}
\end{frame}

\begin{frame}{Razão de chances (OR)}
  \begin{block}{Definição}
    Razão de chances (odds ratio, OR) é a razão entre as chances do grupo exposto/tratado e o grupo não exposto/não tratado.
  \end{block}
  \begin{itemize}
  \item OR $> 1 \Rightarrow$ exposição aumenta chances do desfecho
  \item OR $< 1 \Rightarrow$ exposição diminui chances do desfecho
  \item OR $\approx 1 \Rightarrow$ exposição não afeta o desfecho
  \end{itemize}
  \begin{exampleblock}{No exemplo, a progressão foi:}
    \begin{itemize}
    \item 19\% sob AZT e 39\% sob placebo
    \item OR = 0.19/0.39 = 0.49 (interprete)
    \item<2-> IC 95\%: [0.36, 0.67] (interprete)
    \end{itemize}
  \end{exampleblock}
\end{frame}

\begin{frame}{Razão de Chances (OR)}
  \begin{exampleblock}{Resultados}
    \begin{itemize}
      \footnotesize
    \item OR = 0.49
    \item IC 95\%: [0.36, 0.67]
    \end{itemize}
  \end{exampleblock}
  \begin{center}
    Interpretação semelhante à RR.
  \end{center}
  \begin{exampleblock}{Conclusão}
    ... participantes tratados com AZT têm 49\% (IC: 36\% a 67\%)menos chances de progressão... em relação ao placebo.
  \end{exampleblock}
\end{frame}

\section{Aprofundamento}

\subsection{Aprofundamento}

\begin{frame}{Aprofundamento}
  \footnotesize
  \begin{block}{Leitura obrigatória}
    \begin{itemize}
    \item Capítulo 8. Pular as seções:
      \begin{itemize}
        \scriptsize
      \item RR de estudos de sobrevivência
      \item Calculando os ICs.
      \end{itemize}
    \item Capítulo 9. Pular a seção:
      \begin{itemize}
        \scriptsize
      \item Cálculo do IC da OR.
      \end{itemize}
    \end{itemize}
    \begin{block}{}
      Ler \alert{atentamente} a discussão sobre quando usar RR e quando usar OR!
    \end{block}
  \end{block}
\end{frame}

\begin{frame}{Aprofundamento}
  \begin{block}{Leitura recomendada}
    \footnotesize
    \begin{itemize}
    \item REIS, I. A.; REIS, E. A. (2001) {\bf Associação entre Variáveis Qualitativas}, Relatório Técnico RTE-05/2001 Dept Estatística UFMG
      \begin{itemize}
        \scriptsize
      \item Seção {\em Medidas de Associação entre Variáveis Qualitativas} (pp. 20)
      \end{itemize}
    \end{itemize}
  \end{block}
\end{frame}

\end{document}
