\everymath{\displaystyle}
\documentclass{beamer}
% \documentclass[handout]{beamer}

%\usepackage[pdftex]{color,graphicx}
\usepackage{amsmath,amssymb,amsfonts}

\mode<presentation>
{
  % \usetheme{Darmstadt}
  % \usetheme[hideothersubsections]{Hannover}
  % \usetheme[hideothersubsections]{Goettingen}
  \usetheme[hideothersubsections, right]{Berkeley}

  \usecolortheme{seahorse}
  % \usecolortheme{dolphin}
  \usecolortheme{rose}
  % \usecolortheme{orchid}

  \useinnertheme[shadow]{rounded}

  \setbeamercovered{transparent}
  % or whatever (possibly just delete it)
}

\mode<handout>{
  \setbeamercolor{background canvas}{bg=black!5}
  \usepackage{pgfpages}
  \pgfpagesuselayout{4 on 1}[a4paper,border shrink=5mm, landscape]
}

\usepackage[brazilian]{babel}
% or whatever

% \usepackage[latin1]{inputenc}
\usepackage[utf8]{inputenc}
% or whatever

\usepackage{times}
% \usepackage[T1]{fontenc}
% Or whatever. Note that the encoding and the font should match. If T1
% does not look nice, try deleting the line with the fontenc.


\title%[Estatística Descritiva I] % (optional, use only with long paper titles)
{Estatística Descritiva I}

\subtitle
{Definições e Distribuições de Frequências} % (optional)

\author%[] % (optional, use only with lots of authors)
{Felipe Figueiredo}% \and S.~Another\inst{2}}
% - Use the \inst{?} command only if the authors have different
%   affiliation.

\institute[INTO] % (optional, but mostly needed)
{
Instituto Nacional de Traumatologia e Ortopedia}
  % \inst{1}%
  % Department of Computer Science\\
  % University of Somewhere
  % \and
  % \inst{2}%
  % Department of Theoretical Philosophy\\
  % University of Elsewhere}
% - Use the \inst command only if there are several affiliations.
% - Keep it simple, no one is interested in your street address.

\date%[Março de 2015] % (optional)
{}

%\subject{Talks}
% This is only inserted into the PDF information catalog. Can be left
% out. 



% If you have a file called "university-logo-filename.xxx", where xxx
% is a graphic format that can be processed by latex or pdflatex,
% resp., then you can add a logo as follows:

\pgfdeclareimage[height=1.6cm]{university-logo}{../logo}
\logo{\pgfuseimage{university-logo}}



% Delete this, if you do not want the table of contents to pop up at
% the beginning of each subsection:
\AtBeginSubsection[]
{
  \begin{frame}<beamer>{Sumário}
    \tableofcontents[currentsection,currentsubsection]
  \end{frame}
}


% If you wish to uncover everything in a step-wise fashion, uncomment
% the following command: 

% \beamerdefaultoverlayspecification{<+->}


\begin{document}

\begin{frame}
  \titlepage
\end{frame}

\begin{frame}{Sumário}
  \tableofcontents
  % You might wish to add the option [pausesections]
\end{frame}


% \section{Análise Descritiva}

% \begin{frame}{Análise Descritiva}
%   \begin{itemize}
%   \item Distribuição de Frequências
%   \item Medidas de Tendência Central
%   \item Medidas de Dispersão
%   \item Medidas de Posição
%   \end{itemize}

% \end{frame}


\section{Tipos de Variáveis}

\begin{frame}{Tipos de Variáveis}
  \begin{definition}
    Variável \alert{dependente} (ou resposta) é a variável a ser
    explicada no estudo.
  \end{definition}
  \begin{definition}
    Variável \alert{independente} (ou explanatória) é a variável que
    serve de suporte na explicação da variabilidade da variável
    resposta.
  \end{definition}
\end{frame}

\begin{frame}{Tipos de Variáveis}
Variáveis podem ser classificadas em duas principais categorias
  \begin{itemize}
  \item Qualitativas (categóricas)
  \item Quantitativas (numéricas)
  \end{itemize}
  \begin{example}
    Pressão sistólica (mmHg), altura (cm), sexo (M ou F), grau de
    satisfação com atendimento médico (nota de 1 a 5), perímetro
    abdominal (cm), contagem de leucócitos, número de pessoas na
    família, cor da pele (branco, negro, pardo), etc.
  \end{example}

\end{frame}

\subsection{Variáveis Qualitativas}

\begin{frame}{Variáveis qualitativas}
Variáveis qualitativas se subdividem em
  \begin{itemize}
  \item<1-2> Nominais
  % \begin{example}
  %   sexo, cor da pele
  % \end{example}

  \item<3-4> Ordinais
  \end{itemize}

  % \begin{example}
  %   satisfação com atendimento médico (nota de 1 a 5)
  % \end{example}

  \begin{example}
    Pressão sistólica (mmHg), altura (cm), \only<2>\alert{sexo (M ou
      F)}, \only<4>\alert{grau de satisfação com atendimento médico
      (nota de 1 a 5)}, perímetro abdominal (cm), contagem de
    leucócitos, número de pessoas na família, \only<2>\alert{cor da
      pele (branco, negro, pardo)}, etc.
  \end{example}

\end{frame}

\subsection{Variáveis Quantitativas}

\begin{frame}{Variáveis quantitativas}
Variáveis quantitativas se subdividem em
  \begin{itemize}
  \item<1-2> Discretos
  \item<3-4> Contínuos
  \end{itemize}
  \begin{example}
    \only<4>\alert{Pressão sistólica (mmHg)}, \only<4>\alert{altura
      (cm)}, sexo (M ou F), grau de satisfação com atendimento médico
    (nota de 1 a 5), perímetro abdominal (cm), \only<2>\alert{contagem
      de leucócitos}, \only<2>\alert{número de pessoas na família},
    cor da pele (branco, negro, pardo), etc.
  \end{example}
\end{frame}

\section{Tipos de Estudos}
\begin{frame}{Tipos de Estudos}
  Estudos podem ser de dois tipos principais
  \begin{itemize}
  \item Observacionais
  \item Experimentais
  \end{itemize}
\end{frame}

\subsection{Estudos experimentais}

\begin{frame}{Estudos experimentais}
  \begin{itemize}
  \item Testar hipóteses em laboratório
  \item Aleatorização e controle
  \item Comparam tratamentos (e.g. ensaio clínico)
  \end{itemize}
  % \begin{example}
  %   Ensaios clínicos, testes com animais, etc
  % \end{example}
\end{frame}

\begin{frame}{Estudos experimentais}
  \begin{example}
    ``Pesquisadores da Universidade Católica da Coreia testaram com
    sucesso uma substância do veneno da aranha-armadeira, produzida
    com células transgênicas de lagarta, para tratar disfunção erétil,
    em \alert{ratos impotentes}. (...) Para produzir a substância
    testada, a proteína PnTx2-6, os pesquisadores modificaram células
    de lagarta com DNA de aranha. Por fim, descobriram que, sob efeito
    da PnTx2-6, os músculos do corpo cavernoso dos ratos relaxavam,
    permitindo a entrada de sangue e (eureca!) a ereção.'' Super
    Interessante, Fevereiro/2015
  \end{example}
  \end{frame}

\subsection{Estudos observacionais}

\begin{frame}{Estudos observacionais}
  \begin{itemize}
  \item Decidir sobre intervenções em populações
  \item Desenho e controle fogem ao controle do pesquisador
  % \item Questões éticas
 \item Comparam populações (e.g. estudos epidemiológicos)
  \end{itemize}
\end{frame}

\begin{frame}{Estudos observacionais}
  \begin{example}
%    Prevalência de câncer de esôfago, epidemia de obesidade, etc
    ``Um grupo de cientistas da Universidade de Harvard descobriu que
    as bebidas açucaradas industrializadas podem causar 184 mil mortes
    por ano.  (...) Para chegar ao número, a equipe cruzou os dados
    correspondentes ao consumo de refrigerantes e sucos no mundo com
    as mortes por doenças associadas à obesidade. (...) Cerca de 70\%
    das 184 mil mortes são causadas pela \alert{diabetes}. O resto é
    por causa de problemas cardíacos e alguns tipos de câncer.'' Super
    Interessante, Março/2013

  \end{example}
\end{frame}

\begin{frame}{Exercício}
  O estudo abaixo é experimental ou observacional?
  
  \begin{block}{Exercício}
    ``Maconha medicinal não é novidade. A erva já é usada mundo afora
    com vários objetivos: diminuir dores, náuseas e alguns efeitos
    secundários de condições como glaucoma, dores nervais e
    câncer. Agora, em meio a diversos debates sobre a droga,
    cientistas descobriram que ela pode retardar ou parar
    completamente a progressão do Mal de Alzheimer. (...)  O estudo
    revelou que pequenas doses de THC (uma substância química presente
    na erva) diminuem a concentração de uma proteína chamada
    beta-amilóide no cérebro. O acúmulo dessa proteína é uma das
    causas do Alzheimer.'' Super Interessante, Setembro/2014
  \end{block}
\end{frame}

\section{Tabelas de Frequências}
\begin{frame}{Tabelas de Frequências de dados}
  \begin{itemize}
  \item Frequência absoluta
  \item Frequência relativa
  \item Frequência acumulada
  \end{itemize}
\end{frame}

\begin{frame}{Tabela de frequências}
  \begin{example}
    Construir uma tabela de distribuições de frequências para o
    seguinte dataset:
    $$ \{ 1,1,1,2,3,3,3,3,3,4,5,5 \}$$
    \begin{center}
      \begin{tabular}[h]{|c|c|c|c|c|}
        \hline
        $x_i$ & $F_i$ & $f_i$ & $F_a$ & $f_a$\\
        \hline
        1 & & & & \\
        \hline
        2 & & & & \\
        \hline
        3 & & & & \\
        \hline
        4 & & & & \\
        \hline
        5 & & & & \\
        \hline
        \hline
        Total & & & & \\
        \hline
      \end{tabular}
      Total de dados: N = 12
    \end{center}
  \end{example}
\end{frame}

\subsection{Frequência absoluta}

\begin{frame}{Frequência absoluta}
  \begin{itemize}
  \item A frequência absoluta ($F$) é a simples contagem da ocorrência
    de cada dado
  \item Soma das frequências: tamanho do dataset.
  \end{itemize}
\end{frame}

\begin{frame}{Tabela de frequências}
  \begin{example}
    Construir uma tabela de distribuições de frequências para o
    seguinte dataset:
    $$ \{ 1,1,1,2,3,3,3,3,3,4,5,5 \}$$
    \begin{center}
      \begin{tabular}[h]{|c|c|c|c|c|}
        \hline
        $x_i$ & $F_i$ & $f_i$ & $F_a$ & $f_a$\\
        \hline
        1 & \alert{\only<2>{3}} & & & \\
        \hline
        2 & \alert{\only<2>{1}} & & & \\
        \hline
        3 & \alert{\only<2>{5}} & & & \\
        \hline
        4 & \alert{\only<2>{1}} & & & \\
        \hline
        5 & \alert{\only<2>{2}} & & & \\
        \hline
        \hline
        Total & \alert{\only<2>{12}} & & & \\
        \hline
      \end{tabular}
      Total de dados: N = 12
    \end{center}
  \end{example}
\end{frame}

\subsection{Frequência relativa}
\begin{frame}{Frequência relativa}
  \begin{itemize}
  \item A frequência relativa ($f$ ou $F\%$) é a frequência absoluta
    dividida pela quantidade total de dados.
  \item Frequências relativas facilitam a comparação de frequências
    entre diferentes datasets.
  \item Soma das frequências: $1 = 100\%$
  \end{itemize}
\end{frame}

\begin{frame}{Tabela de frequências}
  \begin{example}
    Construir uma tabela de distribuições de frequências para o
    seguinte dataset:
    $$ \{ 1,1,1,2,3,3,3,3,3,4,5,5 \}$$
    \begin{center}
      \begin{tabular}[h]{|c|c|c|c|c|}
        \hline
        $x_i$ & $F_i$ & $f_i$ & $F_a$ & $f_a$\\
        \hline
        1 & 3 & \alert{\only<2>{0.25}} & & \\
        \hline
        2 & 1 & \alert{\only<2>{0.08}} & & \\
        \hline
        3 & 5 & \alert{\only<2>{0.42}} & & \\
        \hline
        4 & 1 & \alert{\only<2>{0.08}} & & \\
        \hline
        5 & 2 & \alert{\only<2>{0.17}} & & \\
        \hline
        \hline
        Total & 12 & \alert{\only<2>{1}} & & \\
        \hline
      \end{tabular}
      Total de dados: N = 12
    \end{center}
  \end{example}
\end{frame}

\subsection{Frequência acumulada}
\begin{frame}{Frequência acumulada}
  \begin{itemize}
  \item A frequência acumulada mostra a soma gradual das frequências
    de cada dado, em uma tabela ordenada
  \item Absoluta ($F_a$) ou acumulada ($f_a$)
  \end{itemize}
\end{frame}

\begin{frame}{Tabela de frequências}
  \begin{example}
    Construir uma tabela de distribuições de frequências para o
    seguinte dataset:
    $$ \{ 1,1,1,2,3,3,3,3,3,4,5,5 \}$$
    \begin{center}
      \begin{tabular}[h]{|c|c|c|c|c|}
        \hline
        $x_i$ & $F_i$ & $f_i$ & $F_a$ & $f_a$\\
        \hline
        1 & 3 & 0.25 & \alert{\only<2->{3}} & \alert{\only<3>{0.25}}\\
        \hline
        2 & 1 & 0.08 & \alert{\only<2->{4}} & \alert{\only<3>{0.33}}\\
        \hline
        3 & 5 & 0.42 & \alert{\only<2->{9}} & \alert{\only<3>{0.75}}\\
        \hline
        4 & 1 & 0.08 & \alert{\only<2->{10}} & \alert{\only<3>{0.83}}\\
        \hline
        5 & 2 & 0.17 & \alert{\only<2->{12}} & \alert{\only<3>{1}}\\
        \hline
        \hline
        Total & 12 & 1 & \alert{\only<2->{12}} & \alert{\only<3>{1}}\\
        \hline
      \end{tabular}
      Total de dados: N = 12
    \end{center}
  \end{example}
\end{frame}

\subsection{Intervalos de Classes}

\begin{frame}{Intervalos de Classes}
  \begin{itemize}
  \item E quanto às variáveis quantitativas contínuas?
  \item Idem discretas quando numerosas?
    \pause
  \item Agrupa-se os dados em classes
  \end{itemize}
  \begin{example}
    A idade (anos) pode ser agrupada em faixas etárias.
  \end{example}
\end{frame}

\begin{frame}{Intervalos de Classes}

  Considere as seguintes alturas (cm):
$ \{ 165,163,170,175,175,174,171,186,159,176,$
$170,158,165,176,169,173,168,172,162,178 \} $

\centering\begin{tabular}{|c|c|c|c|c|}
  \hline
  Altura (cm) & $F$ & $f$ & $F_a$ & $f_a$ \\
  \hline
  $155 |- 160$ & 2 & 0.10 & 2 & 0.10\\%..
  $160 |- 165$ & 2 & 0.10 & 4 & 0.20\\%..
  $165 |- 170$ & 4 & 0.20 & 8 & 0.40\\%....
  $170 |- 175$ & 6 & 0.30 & 14 & 0.70\\%......
  $175 |- 180$ & 5 & 0.25 & 19 & 0.95\\%.....
  $180 |- 185$ & 0 & 0.00 & 19 & 0.95\\%
  $185 |- 190$ & 1 & 0.05 & 20 & 1\\%.
%  $190 |- 195$ & 1 &  &  & \\%
  \hline
  \hline
  Total & 20 & 1 & = & =\\
  \hline
\end{tabular}

\end{frame}
\subsection{Resumo}

\begin{frame}{Tabelas x Variáveis}
  \begin{itemize}
  \item As tabelas anteriores podem ser construídas para dados com
    ordem intrínseca.
    \begin{itemize}
    \item Qualitativas Ordinais
    \item Discretas
    \item Contínuas
    \end{itemize}
  \item E quanto às variáveis qualitativas nominais?
  \end{itemize}
  
\end{frame}
\end{document}

