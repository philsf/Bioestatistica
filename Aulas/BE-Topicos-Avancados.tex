\everymath{\displaystyle}
\documentclass{beamer}
% \documentclass[handout]{beamer}

%\usepackage[pdftex]{color,graphicx}
\usepackage{amsmath,amssymb,amsfonts}

\mode<presentation>
{
  % \usetheme{Darmstadt}
  % \usetheme[hideothersubsections]{Hannover}
  % \usetheme[hideothersubsections]{Goettingen}
  \usetheme[hideothersubsections, right]{Berkeley}

  \usecolortheme{seahorse}
  % \usecolortheme{dolphin}
  \usecolortheme{rose}
  % \usecolortheme{orchid}

  \useinnertheme[shadow]{rounded}

  % \setbeamercovered{transparent}
  \setbeamercovered{invisible}
  % or whatever (possibly just delete it)
}

\mode<handout>{
  \setbeamercolor{background canvas}{bg=black!5}
  \usepackage{pgfpages}
  \pgfpagesuselayout{4 on 1}[a4paper,border shrink=5mm, landscape]
}

\usepackage[brazilian]{babel}
% or whatever

% \usepackage[latin1]{inputenc}
\usepackage[utf8]{inputenc}
% or whatever

\usepackage{times}
%\usepackage[T1]{fontenc}
% Or whatever. Note that the encoding and the font should match. If T1
% does not look nice, try deleting the line with the fontenc.


\title%[] % (optional, use only with long paper titles)
{}

\subtitle
{} % (optional)

\author%[] % (optional, use only with lots of authors)
{Felipe Figueiredo}% \and S.~Another\inst{2}}
% - Use the \inst{?} command only if the authors have different
%   affiliation.

\institute[INTO] % (optional, but mostly needed)
{
}
  % \inst{1}%
  % Department of Computer Science\\
  % University of Somewhere
  % \and
  % \inst{2}%
  % Department of Theoretical Philosophy\\
  % University of Elsewhere}
% - Use the \inst command only if there are several affiliations.
% - Keep it simple, no one is interested in your street address.

\date%[] % (optional)
{}

% \subject{Talks}
% This is only inserted into the PDF information catalog. Can be left
% out. 



% If you have a file called "university-logo-filename.xxx", where xxx
% is a graphic format that can be processed by latex or pdflatex,
% resp., then you can add a logo as follows:

\pgfdeclareimage[height=1.6cm]{university-logo}{../logo}
\logo{\pgfuseimage{university-logo}}



% Delete this, if you do not want the table of contents to pop up at
% the beginning of each subsection:
\AtBeginSubsection[]
%\AtBeginSection[]
{
  \begin{frame}<beamer>{Sumário}
    \tableofcontents[currentsection,currentsubsection]
  \end{frame}
}


% If you wish to uncover everything in a step-wise fashion, uncomment
% the following command: 

% \beamerdefaultoverlayspecification{<+->}


\begin{document}

\begin{frame}
  \titlepage
\end{frame}

\begin{frame}{Sumário}
  \tableofcontents
  % You might wish to add the option [pausesections]
\end{frame}


%% Template
% \section{}

% \subsection{}

% \begin{frame}{}
%   \begin{itemize}
%   \item 
%   \end{itemize}
% \end{frame}

% \begin{frame}
%   \begin{columns}
%     \begin{column}{5cm}
%     \end{column}
%     \begin{column}{5cm}
%     \end{column}
%   \end{columns}
% \end{frame}

% \begin{frame}{}
%   \includegraphics[height=0.4\textheight]{file1}
%   \includegraphics[height=0.4\textheight]{file2}
%   \includegraphics[height=0.4\textheight]{file3}
%   \begin{figure}
%     \caption{}
%   \end{figure}
% \end{frame}

% \begin{frame}{}
%   \begin{definition}
%   \end{definition}
%   \begin{example}
%   \end{example}
%   \begin{block}{Exercício}
%   \end{block}
% \end{frame}

\section{Comparações múltiplas}

\begin{frame}{Como comparar médias}
  \begin{itemize}
  \item Vimos que o {\bf teste t} pode ser usado para comparar duas médias
  \item Assumindo que atendemos às premissas do teste t, precisamos levar em conta:
    \begin{itemize}
    \item variabilidade dos grupos
    \item tamanho do estudo (n)
    \end{itemize}
  \end{itemize}
  \begin{block}{Requisitos não óbvios (além das médias)}
    desvio padrão + n = erro padrão
  \end{block}
\end{frame}

\begin{frame}
  \begin{block}{}
    O que é necessário para decidir se 3 (ou mais) grupos possuem médias diferentes?
  \end{block}
\end{frame}

\begin{frame}{Esses 3 grupos têm médias diferentes?}
  \begin{center}
    \includegraphics[height=.9\textheight]{Topicos_adv/cenario1}
  \end{center}
\end{frame}

\begin{frame}{Médias: Placebo: 4.210, Tratamento A: 3.250, Tratamento B: 3.845}
  \begin{center}
    \includegraphics[height=.9\textheight]{Topicos_adv/cenario1_medias}

%    {\tiny Médias: Placebo: 4.210, Tratamento A: 3.250, Tratamento B: 3.845}
  \end{center}
\end{frame}

\begin{frame}{E estes 3 grupos?}
  \begin{center}
    \includegraphics[height=.9\textheight]{Topicos_adv/cenario2}
  \end{center}
\end{frame}

\begin{frame}{Médias: Placebo: 4.559, Tratamento A: 5.855, Tratamento B: 5.928}
  \begin{center}
    \includegraphics[height=.9\textheight]{Topicos_adv/cenario2_medias}

%    {\tiny Médias: Placebo: 4.559, Tratamento A: 5.855, Tratamento B: 5.928}
  \end{center}
\end{frame}

\begin{frame}{Comparação entre 3 (ou mais) grupos}
  \begin{block}{Abordagem mais simples}
    Uma ideia seria usar o teste t três vezes, comparando os grupos aos pares.

    \bigskip
    Testar se há diferenças significativas, e seus respectivos tamanhos.
  \end{block}

  \begin{exampleblock}{Exemplo}
    \begin{enumerate}
    \item Placebo x Tratamento A
    \item Placebo x Tratamento B
    \item Tratamento A x Tratamento B
    \end{enumerate}
  \end{exampleblock}
\end{frame}

\begin{frame}{Exemplo 1}
  \begin{columns}
    \begin{column}{5cm}
      \begin{exampleblock}{P-valores dos 3 testes t}
        \tiny
        \begin{enumerate}
        \item Placebo x Trat. A $\Rightarrow p=0.02652$
        \item Placebo x Trat. B $\Rightarrow p=0.4331$
        \item Trat. A x Trat. B $\Rightarrow p=0.09686$
        \end{enumerate}
      \end{exampleblock}
      \begin{exampleblock}{Pergunta}
        \small
        Qual é a conclusão correta quanto à comparação destes grupos?
      \end{exampleblock}
    \end{column}
    \begin{column}{5cm}
      \includegraphics[width=\textwidth]{Topicos_adv/cenario1}
    \end{column}
  \end{columns}
\end{frame}

\begin{frame}{Exemplo 2}
  \begin{columns}
    \begin{column}{5cm}
      \begin{exampleblock}{P-valores dos 3 testes t}
        \tiny
        \begin{enumerate}
        \item Placebo x Trat. A $\Rightarrow p=0.0399$
        \item Placebo x Trat. B $\Rightarrow p=0.02235$
        \item Trat. A x Trat. B $\Rightarrow p=0.8432$
        \end{enumerate}
      \end{exampleblock}
      \begin{exampleblock}{Pergunta}
        \small
        E no segundo cenário?

        Os tratamentos são diferentes do placebo?
        E entre si?
      \end{exampleblock}
    \end{column}
    \begin{column}{5cm}
      \includegraphics[width=\textwidth]{Topicos_adv/cenario2}
    \end{column}
  \end{columns}
\end{frame}

\section[ANOVA]{Análise de Variância (ANOVA)}

\subsection{Comparação de 3 (ou mais) grupos}

% \begin{frame}{3 grupos - cenário 1}
%   \begin{center}
%     \includegraphics[height=.9\textheight]{Topicos_adv/cenario1}
%   \end{center}
% \end{frame}

% \begin{frame}{Médias: Placebo: 4.210, Tratamento A: 3.250, Tratamento B: 3.845}
%   \begin{center}
%     \includegraphics[height=.9\textheight]{Topicos_adv/cenario1_medias}

% %    {\tiny Médias: Placebo: 4.210, Tratamento A: 3.250, Tratamento B: 3.845}
%   \end{center}
% \end{frame}

% \begin{frame}{3 grupos - cenário 2}
%   \begin{center}
%     \includegraphics[height=.9\textheight]{Topicos_adv/cenario2}
%   \end{center}
% \end{frame}

% \begin{frame}{Médias: Placebo: 4.559, Tratamento A: 5.855, Tratamento B: 5.928}
%   \begin{center}
%     \includegraphics[height=.9\textheight]{Topicos_adv/cenario2_medias}

% %    {\tiny Médias: Placebo: 4.559, Tratamento A: 5.855, Tratamento B: 5.928}
%   \end{center}
% \end{frame}

\subsection{O teste F}

\subsection{Pós teste}

\section{Resumo}

\subsection{Resumo}

\begin{frame}{Resumo}
  \begin{itemize}
  \item
  \end{itemize}
\end{frame}

\begin{frame}{Leitura pós-aula e exercícios selecionados}
  \begin{block}{Leitura obrigatória}
    \begin{itemize}
    \item Capítulo 13
    \item 
    \end{itemize}
  \end{block}
  \begin{block}{Exercícios}
    Capítulo 19, problemas: todos menos o problema 5.
  \end{block}
  \begin{block}{Leitura recomendada}
  \end{block}
\end{frame}

\end{document}
