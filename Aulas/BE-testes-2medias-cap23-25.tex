\everymath{\displaystyle}
\documentclass{beamer}
% \documentclass[handout]{beamer}

%\usepackage[pdftex]{color,graphicx}
\usepackage{amsmath,amssymb,amsfonts}

\mode<presentation>
{
  % \usetheme{Darmstadt}
  % \usetheme[hideothersubsections]{Hannover}
  % \usetheme[hideothersubsections]{Goettingen}
  \usetheme[hideothersubsections, right]{Berkeley}

  \usecolortheme{seahorse}
  % \usecolortheme{dolphin}
  \usecolortheme{rose}
  % \usecolortheme{orchid}

  \useinnertheme[shadow]{rounded}

  % \setbeamercovered{transparent}
  \setbeamercovered{invisible}
  % or whatever (possibly just delete it)
}

\mode<handout>{
  \setbeamercolor{background canvas}{bg=black!5}
  \usepackage{pgfpages}
  \pgfpagesuselayout{4 on 1}[a4paper,border shrink=5mm, landscape]
}

\usepackage[brazilian]{babel}
% or whatever

% \usepackage[latin1]{inputenc}
\usepackage[utf8]{inputenc}
% or whatever

\usepackage{times}
%\usepackage[T1]{fontenc}
% Or whatever. Note that the encoding and the font should match. If T1
% does not look nice, try deleting the line with the fontenc.


\title%[] % (optional, use only with long paper titles)
{Comparação de dois grupos (quantitativo)}

\subtitle
{Testes para médias} % (optional)

\author%[] % (optional, use only with lots of authors)
{Felipe Figueiredo}% \and S.~Another\inst{2}}
% - Use the \inst{?} command only if the authors have different
%   affiliation.

\institute[INTO] % (optional, but mostly needed)
{Instituto Nacional de Traumatologia e Ortopedia
}
  % \inst{1}%
  % Department of Computer Science\\
  % University of Somewhere
  % \and
  % \inst{2}%
  % Department of Theoretical Philosophy\\
  % University of Elsewhere}
% - Use the \inst command only if there are several affiliations.
% - Keep it simple, no one is interested in your street address.

\date%[] % (optional)
{}

% \subject{Talks}
% This is only inserted into the PDF information catalog. Can be left
% out. 



% If you have a file called "university-logo-filename.xxx", where xxx
% is a graphic format that can be processed by latex or pdflatex,
% resp., then you can add a logo as follows:

\pgfdeclareimage[height=1.6cm]{university-logo}{../logo}
\logo{\pgfuseimage{university-logo}}



% Delete this, if you do not want the table of contents to pop up at
% the beginning of each subsection:
\AtBeginSubsection[]
%\AtBeginSection[]
{
  \begin{frame}<beamer>{Sumário}
    \tableofcontents[currentsection,currentsubsection]
  \end{frame}
}


% If you wish to uncover everything in a step-wise fashion, uncomment
% the following command: 

% \beamerdefaultoverlayspecification{<+->}


\begin{document}

\begin{frame}
  \titlepage
\end{frame}

\begin{frame}{Sumário}
  \tableofcontents
  % You might wish to add the option [pausesections]
\end{frame}


%% Template
% \section{}

% \subsection{}

% \begin{frame}{}
%   \begin{itemize}
%   \item 
%   \end{itemize}
% \end{frame}

% \begin{frame}
%   \begin{columns}
%     \begin{column}{5cm}
%     \end{column}
%     \begin{column}{5cm}
%     \end{column}
%   \end{columns}
% \end{frame}

% \begin{frame}{}
%   \includegraphics[height=0.4\textheight]{file1}
%   \includegraphics[height=0.4\textheight]{file2}
%   \includegraphics[height=0.4\textheight]{file3}
%   \begin{figure}
%     \caption{}
%   \end{figure}
% \end{frame}

% \begin{frame}{}
%   \begin{definition}
%   \end{definition}
%   \begin{example}
%   \end{example}
%   \begin{block}{Exercício}
%   \end{block}
% \end{frame}

\begin{frame}{Revisão: hipótese nula}
  \begin{block}{Conceito da hipótese nula}
    A hipótese de que não há efeito no tratamento.

    \bigskip
    O objetivo do estudo é providenciar evidências suficientes para rejeitar esta hipótese, provando assim a eficácia do tratamento.
  \end{block}
  \begin{exampleblock}{Exemplo}
    Hipótese do estudo: um certo tratamento de fisioterapia diminui o tempo de recuperação após uma artroplasia total do joelho.

    \bigskip
    Hipótese nula: não há alteração no tempo de recuperação.
  \end{exampleblock}
\end{frame}

\begin{frame}{Revisão: p-valor}
  \begin{block}{Conceito do p-valor}
    Assumindo que não há efeito real (hipótese nula), e você observou uma aparente diferença... qual é a probabilidade de você ter observado essa diferença ao acaso?
  \end{block}
\end{frame}

\begin{frame}{Revisão: p-valor}
\begin{block}{Interpretação do p-valor}
  \begin{itemize}
  \item Um valor pequeno para o p-valor (tipicamente $p \le 0.05$)
    representa forte evidência para rejeitar a hipótese nula, então
    deve-se rejeitá-la.
  \item Um valor alto para o p-valor (tipicamente $p \ge 0.05$)
    representa pouca evidência contra a hipótese nula, então não se
    deve rejeitá-la
  \item Um valor próximo do ponto de corte ($0.05$) é considerado
    marginal, portanto ``qualquer decisão pode ser tomada''. Sempre
    apresente seu p-valor para que o leitor possa tirar suas próprias
    conclusões.
  \end{itemize}
\end{block}
Fonte: Rumsey, D. (Statistics for Dummies, 2nd ed.)
\end{frame}

\begin{frame}{Grupos pareados}
Quando faz sentido parear indivíduos de dois grupos?
  \begin{itemize}
  \item Mensurar o \alert{mesmo} individuo antes e depois do procedimento
  \item Recrutamento aos pares, quando o par tem a(o) mesma(o)
    \begin{itemize}
    \item idade/faixas etária
    \item região demográfica
    \item diagnóstico
    \end{itemize}
  \item irmãos, pai/filho
  \item lateralidade (tratamento = lado E, controle = lado D)
  \end{itemize}
\end{frame}

\end{document}
