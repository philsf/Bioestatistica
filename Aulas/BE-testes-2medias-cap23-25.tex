\everymath{\displaystyle}
\documentclass{beamer}
% \documentclass[handout]{beamer}

%\usepackage[pdftex]{color,graphicx}
\usepackage{amsmath,amssymb,amsfonts}

\mode<presentation>
{
  % \usetheme{Darmstadt}
  % \usetheme[hideothersubsections]{Hannover}
  % \usetheme[hideothersubsections]{Goettingen}
  \usetheme[hideothersubsections, right]{Berkeley}

  \usecolortheme{seahorse}
  % \usecolortheme{dolphin}
  \usecolortheme{rose}
  % \usecolortheme{orchid}

  \useinnertheme[shadow]{rounded}

  % \setbeamercovered{transparent}
  \setbeamercovered{invisible}
  % or whatever (possibly just delete it)
}

\mode<handout>{
  \setbeamercolor{background canvas}{bg=black!5}
  \usepackage{pgfpages}
  \pgfpagesuselayout{4 on 1}[a4paper,border shrink=5mm, landscape]
}

\usepackage[brazilian]{babel}
% or whatever

% \usepackage[latin1]{inputenc}
\usepackage[utf8]{inputenc}
% or whatever

\usepackage{times}
%\usepackage[T1]{fontenc}
% Or whatever. Note that the encoding and the font should match. If T1
% does not look nice, try deleting the line with the fontenc.


\title%[] % (optional, use only with long paper titles)
{Comparação de dois grupos (quantitativo)}

\subtitle
{Testes para médias} % (optional)

\author%[] % (optional, use only with lots of authors)
{Felipe Figueiredo}% \and S.~Another\inst{2}}
% - Use the \inst{?} command only if the authors have different
%   affiliation.

\institute[INTO] % (optional, but mostly needed)
{Instituto Nacional de Traumatologia e Ortopedia
}
  % \inst{1}%
  % Department of Computer Science\\
  % University of Somewhere
  % \and
  % \inst{2}%
  % Department of Theoretical Philosophy\\
  % University of Elsewhere}
% - Use the \inst command only if there are several affiliations.
% - Keep it simple, no one is interested in your street address.

\date%[] % (optional)
{}

% \subject{Talks}
% This is only inserted into the PDF information catalog. Can be left
% out. 



% If you have a file called "university-logo-filename.xxx", where xxx
% is a graphic format that can be processed by latex or pdflatex,
% resp., then you can add a logo as follows:

\pgfdeclareimage[height=1.6cm]{university-logo}{../logo}
\logo{\pgfuseimage{university-logo}}



% Delete this, if you do not want the table of contents to pop up at
% the beginning of each subsection:
\AtBeginSubsection[]
%\AtBeginSection[]
{
  \begin{frame}<beamer>{Sumário}
    \tableofcontents[currentsection,currentsubsection]
  \end{frame}
}


% If you wish to uncover everything in a step-wise fashion, uncomment
% the following command: 

% \beamerdefaultoverlayspecification{<+->}


\begin{document}

\begin{frame}
  \titlepage
\end{frame}

\begin{frame}{Sumário}
  \tableofcontents
  % You might wish to add the option [pausesections]
\end{frame}


%% Template
% \section{}

% \subsection{}

% \begin{frame}{}
%   \begin{itemize}
%   \item 
%   \end{itemize}
% \end{frame}

% \begin{frame}
%   \begin{columns}
%     \begin{column}{5cm}
%     \end{column}
%     \begin{column}{5cm}
%     \end{column}
%   \end{columns}
% \end{frame}

% \begin{frame}{}
%   \includegraphics[height=0.4\textheight]{file1}
%   \includegraphics[height=0.4\textheight]{file2}
%   \includegraphics[height=0.4\textheight]{file3}
%   \begin{figure}
%     \caption{}
%   \end{figure}
% \end{frame}

% \begin{frame}{}
%   \begin{definition}
%   \end{definition}
%   \begin{example}
%   \end{example}
%   \begin{block}{Exercício}
%   \end{block}
% \end{frame}

\section{Revisão}

\begin{frame}{Revisão: hipótese nula}
  \begin{block}{Conceito da hipótese nula}
    {\bf A hipótese de que não há efeito no tratamento.}

    \bigskip
    O objetivo do estudo é providenciar evidências suficientes para rejeitar esta hipótese, provando assim a eficácia do tratamento.
  \end{block}
  \begin{exampleblock}{Exemplo}
    {\bf Hipótese do estudo:} um certo tratamento de fisioterapia diminui o tempo de recuperação após uma artroplasia total do joelho.

    \bigskip
    {\bf Hipótese nula: não há alteração no tempo de recuperação.}
  \end{exampleblock}
\end{frame}

\begin{frame}{Revisão: p-valor}
  \begin{block}{Conceito do p-valor}
    Assumindo que não há efeito real (hipótese nula), e você observou uma aparente diferença... qual é a probabilidade de você ter observado essa diferença ao acaso?
  \end{block}
\end{frame}

\begin{frame}{Revisão: p-valor}
\begin{block}{Interpretação do p-valor}
  \begin{itemize}
  \item Um valor pequeno para o p-valor (tipicamente $p \le 0.05$)
    representa forte evidência para rejeitar a hipótese nula, então
    deve-se rejeitá-la.
  \item Um valor alto para o p-valor (tipicamente $p \ge 0.05$)
    representa pouca evidência contra a hipótese nula, então não se
    deve rejeitá-la
  \item Um valor próximo do ponto de corte ($0.05$) é considerado
    marginal, portanto ``qualquer decisão pode ser tomada''. Sempre
    apresente seu p-valor para que o leitor possa tirar suas próprias
    conclusões.
  \end{itemize}
\end{block}
Fonte: Rumsey, D. (Statistics for Dummies, 2nd ed.)
\end{frame}

\section{Testes paramétricos para médias}

\begin{frame}{Testes estatísticos}
Testes estatísticos sempre seguem o mesmo roteiro
  \begin{enumerate}
  \item As estatísticas sumárias são calculadas a partir da amostra
  \item Estas são usadas para calcular uma {\bf estatística de teste}
  \item O valor da estatística de teste é o critério de decisão:
    \begin{itemize}
    \item Pode ser comparado com um valor crítico, da distribuição de probabilidades; OU
    \item {\bf A estatística de teste é usada para o cálculo do p-valor, e este é usado como critério}
    \end{itemize}
  \end{enumerate}
\end{frame}

\begin{frame}{Testes paramétricos}
  \begin{itemize}
  \item Existe uma infinidade de testes estatísticos (cada qual com sua hipótese nula)
  \item São divididos em dois grandes grupos: paramétricos e não paramétricos
  \item {\bf Os testes paramétricos assumem que a amostra vem de uma \alert{distribuição Normal}}
  \item Os testes não-paramétricos não presumem nenhuma forma para a distribuição dos dados
  \end{itemize}
  \begin{block}{Atenção}
    Esta é uma escolha metodológica fundamental na análise, como veremos no futuro.
  \end{block}
\end{frame}

\begin{frame}{Testes paramétricos}
  \begin{itemize}
  \item Os testes paramétricos assumem que a amostra vem de uma \alert{distribuição Normal} \footnote{nunca é demais frisar}
  \item Hoje veremos o {\bf teste t} (de Student), aplicado em duas formas/contextos
  % \item Ele pode ser usado para um ou dois grupos de medições.
  \end{itemize}
\end{frame}

\subsection{Dois grupos independentes}

\begin{frame}{Premissas}
  \begin{itemize}
  \item Os dois grupos foram coletados independentemente (inter-grupo)
  \item Todas as observações em cada grupo são independentes entre si (intra-grupo)
  \item {\bf Todos os dados foram amostrados de populações Normalmente distribuídas (aprox.)}
  \item O DP das duas populações são idênticos \footnote{uma violação desta premissa não é grave - buscar aproximação de Welch.}
  \end{itemize}
\end{frame}

\begin{frame}{Exemplo}
  \begin{exampleblock}{Exemplo 23.2}
    Motulsky, {\em et al.} (1983) investigaram se pessoas com hipertensão tem alteração nos níveis de receptores adrenérgicos $\alpha_2$ em suas plaquetas.
    Selecionaram 18 homens hipertensos, e 17 controles da mesma faixa etária.

    As plaquetas dos hipertensos tiveram 257 $\pm$ 14 receptores por plaqueta (média $\pm$ SEM).

    As plaquetas dos controles tiveram 263 $\pm$ 21 receptores por plaqueta (média $\pm$ SEM).

    \bigskip
    Os autores concluíram que não havia diferença significativa entre as médias dos grupos.
  \end{exampleblock}
\end{frame}

\subsection{Dois grupos pareados}

\begin{frame}{Grupos independentes x pareados}
  \begin{itemize}
  \item Assim como no cálculo de ICs, os grupos de estudo podem ser independentes ou pareados
  \item Quando são independentes, a comparação é entre as médias de ambos os grupos
  \item Quando são pareados, a comparação é entre as diferenças dos pares
  \end{itemize}
\end{frame}


\begin{frame}{Grupos pareados (revisão)}
Quando faz sentido parear indivíduos de dois grupos?
  \begin{itemize}
  \item Mensurar o \alert{mesmo} indivíduo antes e depois do procedimento
  \item Recrutamento aos pares, quando o par tem a(o) mesma(o)
    \begin{itemize}
    \item idade/faixas etária
    \item região demográfica
    \item diagnóstico
    \end{itemize}
  \item irmãos, pai/filho
  \item lateralidade (tratamento = lado E, controle = lado D)
  \end{itemize}
\end{frame}

\begin{frame}{Premissas}
  \begin{itemize}
  \item Os pares amostrados aleatoriamente de uma mesma população (ou representativa)
  \item Os participantes são pareados - o primeiro do grupo A com o primeiro do grupo B, etc.
  \item Cada par é independente de todos os outros
  \item {\bf A distribuição das diferenças, na população, é Normalmente distribuída (aprox.)}
  \end{itemize}
\end{frame}

\subsection{Exercício}

\begin{frame}{Exercício}
  \begin{exampleblock}{Exercício}
    Queremos avaliar a eficiência de uma nova dieta reduzida em
    gordura no tratamento de obesidade. Selecionamos aleatoriamente
    100 pessoas obesas para o grupo 1, que receberão a dieta com pouca
    gordura. Selecionamos outras 100 pessoas obesas para o grupo 2 que
    receberão a mesma quantidade de comida, com proporção normal de
    gordura. Após 4 meses, a perda de peso média no grupo 1 foi de
    9.33 lbs (s=4.72) e no grupo 2 foi de 7.58 lbs (s=3.90). Você acha
    que essa nova dieta é eficaz na perda de peso?
  \end{exampleblock}
  Fonte: Khan Academy
\end{frame}

\begin{frame}[label=perguntas]{Perguntas}
  \begin{itemize}
  \item Para este estudo, qual dos dois testes é o mais apropriado?
  \item Qual é a hipótese nula?
  \item Qual é a hipótese alternativa?
  \item O que você usaria como critério de decisão?
  \item Qual é o resultado?
  \item Qual é a conclusão?
  \item O que significam valores negativos neste caso?
  \end{itemize}
\end{frame}

\begin{frame}{Visualização (independentes)}
  \includegraphics[height=\textheight]{Teste_t/2-amostras-independentes}
\end{frame}

\begin{frame}{Visualização (pareados)}
    \includegraphics[height=\textheight]{Teste_t/2-amostras-pareadas}
  \end{frame}

\againframe{perguntas}

\begin{frame}[fragile]{Saída típica de um programa}
  \begin{block}{Teste t, amostras independentes}
    \begin{verbatim}
	Two Sample t-test

data:  dados$Grupo1 and dados$Grupo2
t = 2.871, df = 198, p-value = 0.004537
alternative hypothesis: true difference 
 in means is not equal to 0
95 percent confidence interval:
 0.5506833 2.9667462
sample estimates:
mean of x mean of y 
 9.334005  7.575291
    \end{verbatim}
  \end{block}
\end{frame}

\begin{frame}[fragile]{Saída típica de um programa}
  \begin{block}{Teste t, amostras pareadas}
    \begin{verbatim}
	Paired t-test

data:  dados$Grupo1 and dados$Grupo2
t = 2.9545, df = 99, p-value = 0.003913
alternative hypothesis: true difference 
 in means is not equal to 0
95 percent confidence interval:
 0.5775744 2.9398551
sample estimates:
mean of the differences 
               1.758715
    \end{verbatim}
  \end{block}
\end{frame}

\againframe{perguntas}

\section{Resumo e encerramento}

\begin{frame}{Resumo}
  \begin{itemize}
  \item O teste t é um teste paramétrico (assume dados Normalmente distribuídos)
  \item O teste t para dois grupos independentes assume independência inter- e intra-grupo
  \item O teste t para dois grupos pareados assume independência entre os pares
  % \item 
  \end{itemize}
\end{frame}

\begin{frame}{Leitura pós-aula e exercícios selecionados}
  \begin{block}{Leitura obrigatória}
    \begin{itemize}
    \item Capítulo 23, pular as seções: Cálculo do teste t em uma tabela, Cálculo do poder.
    \item Capítulo 25, pular as seções: Teste t de uma razão, Teste de Wilcoxon
    \end{itemize}
  \end{block}
  \begin{block}{Exercícios}
    Não há exercícios.
  \end{block}
  \begin{block}{Leitura recomendada}
    Capítulo 25: seção teste t de uma razão (para projetos experimentais)
  \end{block}
\end{frame}

\end{document}
