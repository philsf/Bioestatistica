\everymath{\displaystyle}
\documentclass{beamer}
% \documentclass[handout]{beamer}

%\usepackage[pdftex]{color,graphicx}
\usepackage{amsmath,amssymb,amsfonts}

\mode<presentation>
{
  % \usetheme{Darmstadt}
  % \usetheme[hideothersubsections]{Hannover}
  % \usetheme[hideothersubsections]{Goettingen}
  \usetheme[hideothersubsections, right]{Berkeley}

  \usecolortheme{seahorse}
  % \usecolortheme{dolphin}
  \usecolortheme{rose}
  % \usecolortheme{orchid}

  \useinnertheme[shadow]{rounded}

  % \setbeamercovered{transparent}
  \setbeamercovered{invisible}
  % or whatever (possibly just delete it)
}

\mode<handout>{
  \setbeamercolor{background canvas}{bg=black!5}
  \usepackage{pgfpages}
  \pgfpagesuselayout{4 on 1}[a4paper,border shrink=5mm, landscape]
}

\usepackage[brazilian]{babel}
% or whatever

% \usepackage[latin1]{inputenc}
\usepackage[utf8]{inputenc}
% or whatever

\usepackage{times}
%\usepackage[T1]{fontenc}
% Or whatever. Note that the encoding and the font should match. If T1
% does not look nice, try deleting the line with the fontenc.

\usepackage{ulem}

\title%[] % (optional, use only with long paper titles)
{Regressão Linear Simples}

\subtitle
{Modelos com desfecho contínuo} % (optional)

\author%[] % (optional, use only with lots of authors)
{Felipe Figueiredo}% \and S.~Another\inst{2}}
% - Use the \inst{?} command only if the authors have different
%   affiliation.

\institute[INTO] % (optional, but mostly needed)
{Instituto Nacional de Traumatologia e Ortopedia
}
  % \inst{1}%
  % Department of Computer Science\\
  % University of Somewhere
  % \and
  % \inst{2}%
  % Department of Theoretical Philosophy\\
  % University of Elsewhere}
% - Use the \inst command only if there are several affiliations.
% - Keep it simple, no one is interested in your street address.

\date%[] % (optional)
{}

% \subject{Talks}
% This is only inserted into the PDF information catalog. Can be left
% out. 



% If you have a file called "university-logo-filename.xxx", where xxx
% is a graphic format that can be processed by latex or pdflatex,
% resp., then you can add a logo as follows:

\pgfdeclareimage[height=1.6cm]{university-logo}{../logo}
\logo{\pgfuseimage{university-logo}}



% Delete this, if you do not want the table of contents to pop up at
% the beginning of each subsection:
\AtBeginSubsection[]
%\AtBeginSection[]
{
  \begin{frame}<beamer>{Sumário}
    \tableofcontents[currentsection,currentsubsection]
  \end{frame}
}


% If you wish to uncover everything in a step-wise fashion, uncomment
% the following command: 

% \beamerdefaultoverlayspecification{<+->}


\begin{document}

\begin{frame}
  \titlepage
\end{frame}

\begin{frame}{Sumário}
  \tableofcontents
  % You might wish to add the option [pausesections]
\end{frame}


%% Template
% \section{}

% \subsection{}

% \begin{frame}{}
%   \begin{itemize}
%   \item 
%   \end{itemize}
% \end{frame}

% \begin{frame}
%   \begin{columns}
%     \begin{column}{5cm}
%     \end{column}
%     \begin{column}{5cm}
%     \end{column}
%   \end{columns}
% \end{frame}

% \begin{frame}{}
%   \includegraphics[height=0.4\textheight]{file1}
%   \includegraphics[height=0.4\textheight]{file2}
%   \includegraphics[height=0.4\textheight]{file3}
%   \begin{figure}
%     \caption{}
%   \end{figure}
% \end{frame}

% \begin{frame}{}
%   \begin{definition}
%   \end{definition}
%   \begin{example}
%   \end{example}
%   \begin{block}{Exercício}
%   \end{block}
% \end{frame}

\section{Discussão da aula passada}

\subsection{Discussão da aula passada}

\begin{frame}{Discussão da aula passada}
  \begin{block}{}
    Discussão da leitura obrigatória da aula passada
  \end{block}
\end{frame}

\section{Modelagem}

\subsection{Modelos em geral}

\begin{frame}{Modelos}
  \begin{block}{Definição}
    \small
    Versão simplificada da realidade, adequada ao fim pretendido.
  \end{block}
  \bigskip
  \small
  Modelos servem para:
  \begin{itemize}
    \footnotesize
  \item representar fenômenos, experimentos, dados, etc. de \alert{forma tratável};
  \item avaliar cenários controlados, menos complexos que a realidade;
  \item extrapolar resultados e conclusões.
  \end{itemize}
\end{frame}

\begin{frame}{Modelos}
  \centering
  \includegraphics[height=\textheight]{Cap18-19/gi}
\end{frame}

\begin{frame}{Modelos animais}
  \centering
  \includegraphics[width=\textwidth]{Cap18-19/Fatmouse}
\end{frame}

\begin{frame}{Modelos animais}
  \centering
  \includegraphics[width=\textwidth]{Cap18-19/GFP_hiir}
\end{frame}

\subsection{Trailer}

\begin{frame}{Modelos estatísticos}
  \begin{itemize}
    \footnotesize
  \item Distribuições de probabilidade servem como modelo para a distribuição dos dados (teórico x empírico)
    \bigskip
  \item Modelos de regressão servem como um {\it framework} para testar hipóteses {\bf específicas} sobre a relação presumida entre variáveis
  \end{itemize}
  \vfill
  \begin{block}{Modelo de regressão}
    Formulação explícita de uma hipótese sobre a associação entre o desfecho (contínuo, neste contexto) e o preditor
  \end{block}
\end{frame}

\begin{frame}{Modelos estatísticos}
  \begin{block}{Modelo explicativo/explanatório}
    \small
    Verificação ou teste de hipóteses sobre a relação entre as variáveis avaliadas.
  \end{block}
  \bigskip
  \begin{block}{Modelo preditivo}
    \small
    Estimativa do resultado esperado, mesmo para dados que não foram testados...

    \bigskip
    \footnotesize
    ... restrito ao intervalo testado.
  \end{block}
\end{frame}

\begin{frame}{Para todos os gostos...}
  \begin{center}
    \includegraphics[height=\textheight]{Cap18-19/modelos-table1}
  \end{center}
\end{frame}

\begin{frame}{\scriptsize Decaimento de anticorpos de neonatos recebidos da mãe}
  \begin{block}{\scriptsize Modelo linear -- polinômio de grau 1}
    \begin{center}
      \includegraphics[width=.3\textwidth]{Cap18-19/cinetica-dengue2-poly1}
    \end{center}
  \end{block}
  \begin{block}{\scriptsize Modelos polinomiais de grau 2 e 3}
    \begin{center}
      \includegraphics[width=.3\textwidth]{Cap18-19/cinetica-dengue2-poly2}
      \includegraphics[width=.3\textwidth]{Cap18-19/cinetica-dengue2-poly3}
    \end{center}
  \end{block}
  {\hfill \scriptsize Tese Doutorado Ana Claudia Duarte -- IOC/Fiocruz 2017}
\end{frame}

\begin{frame}{\small E você pensando...}
    \begin{center}
      \includegraphics[height=.8\textheight]{Cap18-19/bhaskara}
    \end{center}
\end{frame}

\begin{frame}{\scriptsize Modelos dose-resposta}
  \begin{block}{\scriptsize Modelo de regressão logística 4 parâmetros}
    \begin{columns}
      \begin{column}{5cm}
        \begin{center}
          \includegraphics[width=.5\textwidth]{Cap18-19/Dose-response-Hill-eq}
        \end{center}
      \end{column}
      \begin{column}{5cm}
        \tiny

        $$\hat{Y} = a + \frac{b-a}{\left[ 1 + \left(\frac{c}{X}\right)\right]}$$
      \end{column}
    \end{columns}
    {\hfill \tiny Gadagkar, Call, 2015; J. Pharmacol. Toxicol. Methods}
  \end{block}
  \begin{block}{\scriptsize Aplicações (EC50, IC50, ED50, TD50, LD50, ...)}
    \begin{center}
      \includegraphics[width=.35\textwidth]{Cap18-19/Dose-response-curve-of-aqueous-garlic-extract-on-viability-of-Mus-musculus-colon}
      \includegraphics[height=.3\textheight]{Cap18-19/Dose-response-curves-and-IC50-values-of-Src-kinase-inhibition-for-baicalin-and-baicalein}
    \end{center}
    {\hfill \tiny [1] Gupta, Lee, 2013;\ \ [2] Jelic, et al., 2016}
  \end{block}
\end{frame}

\begin{frame}
  \begin{center}
    \Large Vamos começar pelo modelo mais simples
  \end{center}
  \vfill
  \hfill \footnotesize (Hoje, apenas desfecho contínuo!)
\end{frame}

\section[Regressão]{Regressão Linear Simples}

\subsection{Introdução}

\begin{frame}{Modelo de regressão linear simples}
  \begin{block}{}
    \small
      Quando os dados indicam uma relação linear, um modelo de regressão pode ser utilizado para quantificar esta relação com uma {\bf reta de regressão}.
    \end{block}
  \begin{exampleblock}{Exemplo: Algumas aplicações}
%    Exemplos de perguntas que podem ser respondidas por tal modelo:

    \begin{itemize}
      \footnotesize
    \item Tendência (``Níveis de insulina em jejum tendem a aumentar com a idade?'')
    \item Ajuste de curva (``Qual é o EC$_{50}$ de uma nova droga?'')
    \item Predição (``Como predizer o risco de infarto do miocárdio, sabendo-se a idade, pressão e nível de colesterol?'')
    \end{itemize}
  \end{exampleblock}
\end{frame}

\begin{frame}{Depois dos comerciais...}
  \begin{center}
    \includegraphics[width=\textwidth]{Cap18-19/bmi-bmd-title}
  \end{center}
\end{frame}

\begin{frame}{Depois dos comerciais...}
  \begin{center}
    \includegraphics[width=1.175\textwidth]{Cap18-19/bmi-bmd-abstract}
  \end{center}
\end{frame}

\begin{frame}{Na prática...}
  \begin{columns}
    \begin{column}{5cm}
      \begin{itemize}
      \item Dados simulados, inspirados no paper.
      \item Existe uma tendência? Ela é linear?
      \item Podemos estimar BMD sabendo o IMC?
      \end{itemize}
    \end{column}
    \begin{column}{5cm}
      \begin{center}
        \includegraphics[width=\textwidth]{Cap18-19/pratica-plot1}
      \end{center}
    \end{column}
  \end{columns}
\end{frame}

\begin{frame}{Quais são as variáveis?}
  \begin{itemize}
  \item Dependente: BMD (contínua)
    \begin{itemize}
      \scriptsize
    \item Sinônimos: desfecho, resposta
    \end{itemize}
  \item Independente: BMI (contínua)
    \begin{itemize}
      \scriptsize
    \item Sinônimos\footnote{\scriptsize Em alguns contextos também covariável/cofator (quando há mais de uma V.I.)}: preditor, fator
    \end{itemize}
  \end{itemize}
  \vfill
  \begin{block}{Esta relação pode ser expressa como}
    \begin{displaymath}
      \text{BMD} \sim \text{BMI}
    \end{displaymath}
  \end{block}
\end{frame}

\begin{frame}{Revisão: equação da reta}
  \begin{itemize}
    \small
  \item A equação de uma reta é definida pela fórmula
    \begin{displaymath}
      \text{BMD} = \alert{a} \times \text{BMI} + \alert{b}
    \end{displaymath}
  \item Duas ``variáveis'' e dois parâmetros
    \begin{itemize}
      \scriptsize
    \item BMD é a variável dependente (dados)
    \item BMI é a variável independente (dados)
    \item \alert{$b$} é o intercepto ({\it intercept})
    \item \alert{$a$} é a inclinação ({\it slope})
    \end{itemize}
  \end{itemize}
  \bigskip
  \begin{block}{\footnotesize Inversão, em relação à matemática básica}<2>
    \begin{itemize}
    \item Note que aqui os ``dados'' já foram coletados (fixos)
    \item Nosso objetivo é estimar os parâmetros da reta \alert{$b$} e \alert{$a$}
    \end{itemize}
  \end{block}
\end{frame}

\begin{frame}{Interpretação dos parâmetros da reta}
  \begin{itemize}
    \small
  \item O intercepto é o valor (hipotético) de BMD quando BMI = 0
  \item A inclinação é quanto BMD altera\footnote{na média!} quando aumentamos o BMI em 1 unidade
  \end{itemize}
  \vfill
  \begin{block}{Atenção}
    Para estas interpretações serem válidas, a relação deve ser linear (proporcional).
  \end{block}
\end{frame}

\begin{frame}{Reta de regressão}
  \begin{block}{Definição}
    \small Uma \alert{reta de regressão} é a reta para a qual a soma
    dos erros quadráticos dos resíduos ($\varepsilon$) é o mínimo.
  \end{block}
  \bigskip
  \uncover<2->{\begin{itemize}
    \scriptsize
  \item Também chamada de reta de melhor ajuste
  \item Minimiza os resíduos (erros aleatórios $\varepsilon$)\footnote{Método dos mínimos quadrados}
  \item Erros aleatórios $\varepsilon$ em torno de zero
  \item Dados observados: Y (desfecho contínuo) e X (preditor)
  \item Parâmetros estimados ($\beta_0$ e $\beta_1$)
  \end{itemize}}
  \vfill
  \begin{block}{Formulação}
    \footnotesize
    \begin{displaymath}
      Y = \beta_0 + \beta_1 X + \varepsilon
    \end{displaymath}
  \end{block}
\end{frame}

\begin{frame}{Resíduos}
  \begin{center}
      \includegraphics[height=0.6\textheight]{Cap18-19/residuos}
  \end{center}

  \begin{block}{Definição}
    \small
    Resíduos são a distância entre o dado observado e a reta.
  \end{block}
\end{frame}

\begin{frame}{Atenção}
  \begin{itemize}
  \item Para muitos testes presume-se que os dados vem de uma distribuição normal
  \item Neste caso, não é necessário que os {\bf dados} sejam normais
  \item {\bf É necessário que os resíduos sejam normais}
  \end{itemize}
\end{frame}

\begin{frame}{Análise de Regressão}
  Para determinar a inclinação e o intercepto, usamos:
  \bigskip
  \begin{itemize}
    \scriptsize
  \item as médias de $X$ e $Y$
  \item as variâncias de $X$ e $Y$
  \item o coeficiente de correlação $r$ entre $X$ e $Y$
  \item o tamanho da amostra $n$
  \item \ldots e algumas operações entre estes termos
  \end{itemize}
\end{frame}

\subsection{A regressão}

\begin{frame}{Exemplo 17.1}
  \begin{exampleblock}{Exemplo 17.1}
    Voltemos ao exemplo de associar a composição lipídica com a sensibilidade a insulina.    
  \end{exampleblock}
  \begin{block}{Pergunta}
    Podemos explicar o ``comportamento'' e a variabilidade da insulina sabendo a composição lipídica?
  \end{block}
\end{frame}

\begin{frame}{Quais são as variáveis?}
  \begin{itemize}
  \item Dependente: insulina (contínua)
  \item Independente: conteúdo lipídico (contínua)
  \end{itemize}
  \vfill
  \begin{block}{Esta relação pode ser expressa como}
    \begin{displaymath}
      \text{insulina} \sim \text{conteúdo lipídico}
    \end{displaymath}
  \end{block}
\end{frame}

\begin{frame}{\small Componentes da regressão linear simples}
  \begin{block}{\footnotesize Versão simplificada (apenas variáveis)}
    \footnotesize
    \begin{displaymath}
      \text{insulina} \sim \text{conteúdo lipídico}
    \end{displaymath}
  \end{block}
  \bigskip
  \bigskip
  \begin{block}{Modelo completo}
    \begin{displaymath}
      \text{insulina} =\beta_0 + \beta_1 \text{(conteúdo lipídico)} + \varepsilon
    \end{displaymath}
  \end{block}
  \vfill
  \small
  Obs: $\varepsilon$ é o erro aleatório residual -- não é explicado pelo modelo
\end{frame}

\begin{frame}{Exemplo 17.1}
  \centering
  \includegraphics[width=.9\textwidth]{Cap18-19/regressao1}

  \vfill
  \hfill \footnotesize Fonte: Motulsky, 1995
\end{frame}

\begin{frame}{Exemplo 17.1}
  \centering
  \includegraphics[width=1.2\textwidth]{Cap18-19/regressao2}
\end{frame}

\begin{frame}{Interpretação}
  \begin{itemize}
  \item O p-valor é significativo.
  \item A inclinação é $\approx \alert{37.2}$
  \item Isto significa que:
  \end{itemize}
  \begin{block}{Interpretação da inclinação}
    \small
    para cada unidade aumentada no \%C20--22...

    \bigskip
    ... teremos um aumento proporcional de aproximadamente 37.2 mg/m$^2$/min na sensibilidade à insulina
  \end{block}
\end{frame}

\begin{frame}{Análise de Regressão}
  \begin{center}
      \includegraphics[height=0.6\textheight]{Cap18-19/residuos2}
  \end{center}

  \small
  Uma forma \alert{simplista} de aferir a qualidade do ajuste do modelo de regressão\footnote{Também chamado de Goodness of Fit} é o Coeficiente de Determinação $r^2$

  {\scriptsize (que corresponde ao quadrado de $r$!)}.
\end{frame}

  % Encontrando a inclinação e o intercepto da reta, podemos estimar
  % $\hat{Y}$ para valores arbitrários de $X$.

\subsection[$R^2$]{Coeficiente de Determinação $r^2$}

\begin{frame}{Coeficiente de Determinação $r^2$}
  \begin{block}{Definição}
    O \alert{coeficiente de determinação} $r^2$ é a relação da
    variação explicada com a variação total.
  \end{block}
  \begin{displaymath}
    r^2 = \frac{\text{variação explicada}}{\text{variação total}}
  \end{displaymath}
  \begin{itemize}
  \item Lembrando: $r^2$ é o quadrado de $r$!
  % \item Este valor tem uma interpretação prática mais fácil, que
  %   veremos em breve
  \end{itemize}
\end{frame}

\begin{frame}{Coeficiente de Determinação $r^2$}
  \begin{block}{}
    \begin{itemize}
      \small
      % \item A variância dos dados pode ser explicada de várias formas
    \item Qual é a porcentagem da variância dos dados pode ser explicada
      pela reta regressora?
      \medskip
    \item O coeficiente $r^2$ é a fração da variância que é
      compartilhada entre $X$ e $Y$.
    \end{itemize}
  \end{block}
    \vfill
    \begin{itemize}
      \small
    \item Obs: Como $r$ está sempre entre -1 e 1
    \begin{itemize}
      \scriptsize
    \item $|r|$ está sempre entre 0 e 1
      \medskip
    \item $r^2$ está sempre entre 0 e 1
    \end{itemize}
  \end{itemize}
\end{frame}

\begin{frame}{Coeficiente de Determinação $r^2$}
  \begin{block}{}
    \centering
    Além disso, $r^2 \le |r|$
  \end{block}
  \bigskip
  \begin{exampleblock}{Por que?}
    \small
    Compare os seguintes números entre 0 e 1:
    \bigskip
    \scriptsize
    \begin{displaymath}
      \frac{1}{2} \text{ e } \left(\frac{1}{2}\right)^2=\frac{1}{4} \Rightarrow
      \frac{1}{4} \le \frac{1}{2}
    \end{displaymath}
    \begin{displaymath}
      \frac{1}{3} \text{ e } \left(\frac{1}{3}\right)^2=\frac{1}{9} \Rightarrow
      \frac{1}{9} \le \frac{1}{3}
    \end{displaymath}
  \end{exampleblock}
\end{frame}

\begin{frame}{Exemplo 17.1}
  \begin{exampleblock}{Exemplo 17.1}
    \small
    Na aula de correlação linear produto-momento de Pearson, vimos que para o exemplo 17.1, $r=0.77$.
    \begin{displaymath}
      r^2 = 0.77^2 = 0.59
    \end{displaymath}
  \end{exampleblock}
        \bigskip
  \begin{exampleblock}{\small Interpretação do Coeficiente de Determinação $r^2$}
    Podemos explicar 59\% da variância da insulina considerando apenas o conteúdo lipídico.
  \end{exampleblock}
\end{frame}

\subsection{Exercício}

\begin{frame}{Na prática...}
  \begin{center}
    \includegraphics[width=\textwidth]{Cap18-19/bmi-bmd-title}
  \end{center}
\end{frame}

\begin{frame}{Na prática...}
  \begin{center}
    \includegraphics[width=1.175\textwidth]{Cap18-19/bmi-bmd-abstract}
  \end{center}
\end{frame}

\begin{frame}{Na prática...}
  \begin{center}
    \includegraphics[height=.9\textheight]{Cap18-19/pratica-plot1}
  \end{center}
\end{frame}

\begin{frame}{Na prática...}
  \begin{center}
    \includegraphics[height=.9\textheight]{Cap18-19/pratica-plot2}
  \end{center}
\end{frame}

\begin{frame}{Na prática...}
  \begin{columns}
    \begin{column}{5cm}
      \begin{itemize}
        \small
      \item \alert{Se o modelo é adequado}, podemos substituir isto...
        \bigskip
      \item<0> ... por isto
        \bigskip
      \item<0> Como saber se o modelo representa bem os dados?
      \end{itemize}
    \end{column}
    \begin{column}{5cm}
      \begin{center}
        \includegraphics[width=1.1\textwidth]{Cap18-19/pratica-plot2}
      \end{center}
    \end{column}
  \end{columns}
\end{frame}

\begin{frame}{Na prática...}
  \begin{columns}
    \begin{column}{5cm}
      \begin{itemize}
        \small
      \item \alert{Se o modelo é adequado}, podemos substituir isto...
        \bigskip
      \item ... por isto
        \bigskip
      \item<2> Como saber se o modelo representa bem os dados?
      \end{itemize}
    \end{column}
    \begin{column}{5cm}
      \begin{center}
        \includegraphics[width=1.1\textwidth]{Cap18-19/pratica-plot4}
      \end{center}
    \end{column}
  \end{columns}
\end{frame}

\begin{frame}{Diagnosticando a regressão}
  \begin{columns}
    \begin{column}{5cm}
      \begin{itemize}
        \small
      \item A dispersão em torno da reta é aprox. aleatória?
        \bigskip
      \item Observe o formato faixa de confiança em torno da reta
        \bigskip
        \scriptsize
      \item A dispersão do desfecho pode ser explicada pela variável independente?
      \item Toda? Parte? Quanta?
      \end{itemize}
    \end{column}
    \begin{column}{5cm}
      \begin{center}
        \includegraphics[width=1.1\textwidth]{Cap18-19/pratica-plot2}
      \end{center}
    \end{column}
  \end{columns}
\end{frame}

\begin{frame}[fragile]{Diagnosticando a regressão}
  \begin{block}{Perguntas}
    \begin{itemize}
      \footnotesize
    \item Os resíduos são aprox. normais?
    \item Quantos \% de variabilidade podem ser explicados pelo modelo?
    \item Qual é o BMD predito para um hipotético BMI = 0?
    \item Quanto o BMD muda, para cada unidade de BMI?
    \end{itemize}
  \end{block}
  \begin{exampleblock}{Saída típica de um programa de análise}
    \tiny
\begin{verbatim}
Residuals:
    Min      1Q  Median      3Q     Max 
-52.097 -13.864   0.762  10.707  58.730 

Coefficients:
            Estimate Std. Error t value Pr(>|t|)    
(Intercept)  98.8176     4.6281   21.35   <2e-16 ***
BMI          -2.9845     0.1846  -16.17   <2e-16 ***
---
Signif. codes:  0 ‘***’ 0.001 ‘**’ 0.01 ‘*’ 0.05 ‘.’ 0.1 ‘ ’ 1

Residual standard error: 20.26 on 198 degrees of freedom
Multiple R-squared:  0.5691,	Adjusted R-squared:  0.5669 
F-statistic: 261.5 on 1 and 198 DF,  p-value: < 2.2e-16
\end{verbatim}
\end{exampleblock}
\end{frame}

\begin{frame}[fragile]{Análise de resíduos}
  \begin{itemize}
  \item Como vimos, os resíduos são erros aleatórios

    ({\footnotesize em torno da reta})
    \bigskip
  \item {\small Erros que não podem ser explicados pelo modelo}
    \bigskip

  \item Devem ser normalmente distribuídos em torno de zero

    ({\footnotesize reta como referência})
  \end{itemize}
  \vfill
  \begin{exampleblock}{Saída típica de um programa de análise}
    \footnotesize
\begin{verbatim}
Residuals:
    Min      1Q  Median      3Q     Max 
-52.097 -13.864   0.762  10.707  58.730 
\end{verbatim}
  \end{exampleblock}
\end{frame}

\begin{frame}{Análise de resíduos}
  \begin{center}
    \Large
    Podemos também verificar esta premissa visualmente
  \end{center}
\end{frame}

\begin{frame}{\footnotesize Análise de resíduos - gráfico de regressão}
  \begin{columns}
    \begin{column}{5cm}
      A distribuição dos resíduos é aprox. Normal?
      \bigskip
      % \bigskip
      \begin{itemize}
        \footnotesize
      \item A dispersão em torno da reta é aprox. aleatória?
        \bigskip
      \item A dispersão dos resíduos aumenta ou diminui ao longo da faixa considerada?
      \end{itemize}
    \end{column}
    \begin{column}{5cm}
      \begin{center}
        \includegraphics[width=1.1\textwidth]{Cap18-19/pratica-plot2}
      \end{center}
    \end{column}
  \end{columns}
\end{frame}

\begin{frame}{\footnotesize Análise de resíduos - gráfico de resíduos}
  \begin{columns}
    \begin{column}{5cm}
      A distribuição dos resíduos é aprox. Normal?
      \bigskip
      % \bigskip
      \begin{itemize}
        \footnotesize
      \item A dispersão em torno de 0 é aprox. aleatória?
        \bigskip
      \item A dispersão dos resíduos aumenta ou diminui ao longo da faixa considerada?
      \end{itemize}
    \end{column}
    \begin{column}{5cm}
      \begin{center}
        \includegraphics[width=1.1\textwidth]{Cap18-19/pratica-plot-resid}
      \end{center}
    \end{column}
  \end{columns}
\end{frame}

\begin{frame}{\footnotesize Análise de resíduos - distribuição dos resíduos}
  \begin{columns}
    \begin{column}{5cm}
      A distribuição dos resíduos é aprox. Normal?
      \bigskip
      % \bigskip
      \begin{itemize}
        \footnotesize
      \item A dispersão em torno de 0 é aprox. aleatória?
        \bigskip
      \item A dispersão dos resíduos aumenta ou diminui ao longo da faixa considerada?
      \end{itemize}
    \end{column}
    \begin{column}{5cm}
      \begin{center}
        \includegraphics[width=1.1\textwidth]{Cap18-19/pratica-hist-resid}
      \end{center}
    \end{column}
  \end{columns}
\end{frame}

\begin{frame}[fragile]{Diagnosticando a regressão}
  \begin{block}{Perguntas}
    \begin{itemize}
      \footnotesize
    \item Os resíduos são aprox. normais?
    \end{itemize}
  \end{block}
  \bigskip
  \begin{exampleblock}{Resposta}
    \begin{itemize}
    \item Sim \hfill \footnotesize (probably...)
    \end{itemize}
  \end{exampleblock}
  \vfill
  \begin{exampleblock}{Saída típica de um programa de análise}
    \scriptsize
\begin{verbatim}
Residuals:
    Min      1Q  Median      3Q     Max 
-52.097 -13.864   0.762  10.707  58.730 
\end{verbatim}
\end{exampleblock}
\end{frame}

\begin{frame}[fragile]{Diagnosticando a regressão}
  \begin{block}{Perguntas}
    \begin{itemize}
      \footnotesize
    \item Quantos \% de variabilidade podem ser explicados pelo modelo?
    \end{itemize}
  \end{block}
  \bigskip
  \begin{exampleblock}{Resposta}
    \begin{itemize}
      \footnotesize 
    \item Podemos explicar R2 = 57\% da variância observada no BMD

      \scriptsize(considerando \alert{apenas} o BMI)
      \bigskip
    \item \scriptsize 43\% são devidos a outros fatores
    \end{itemize}
  \end{exampleblock}
  \vfill
  \begin{exampleblock}{Saída típica de um programa de análise}
    \scriptsize
\begin{verbatim}
Multiple R-squared:  0.5691,	Adjusted R-squared:  0.5669 
\end{verbatim}
\end{exampleblock}
\end{frame}

\begin{frame}
  \begin{center}
    E os parâmetros da reta estimados a partir dos dados?
  \end{center}
\end{frame}

\begin{frame}[fragile]{Diagnosticando a regressão}
  \begin{block}{Perguntas}
    \begin{itemize}
      \footnotesize
    \item Qual é o BMD predito para um hipotético BMI = 0?
    \end{itemize}
  \end{block}
  \bigskip
  \begin{exampleblock}{Resposta}
    \begin{itemize}
    \item BMD = 99 unidades \footnotesize (IC = [89.69, 107.94])
    \end{itemize}
  \end{exampleblock}
  \vfill
  \begin{exampleblock}{Saída típica de um programa de análise}
    \scriptsize
\begin{verbatim}
Coefficients:
            Estimate Std. Error t value Pr(>|t|)    
(Intercept)  98.8176     4.6281   21.35   <2e-16 ***
\end{verbatim}
\end{exampleblock}
\end{frame}

\begin{frame}[fragile]{Diagnosticando a regressão}
  \begin{block}{Perguntas}
    \begin{itemize}
      \footnotesize
    \item Quanto o BMD muda, para cada unidade de BMI?
    \end{itemize}
  \end{block}
  \bigskip
    \begin{exampleblock}{Resposta}
      \begin{itemize}
      \item Decréscimo de 3 unidades de BMD \footnotesize (IC = [-3.35, -2.62])

        \bigskip
        \scriptsize (para cada incremento unitário de BMI)
    \end{itemize}
  \end{exampleblock}
  \vfill
  \begin{exampleblock}{Saída típica de um programa de análise}
    \scriptsize
\begin{verbatim}
Coefficients:
            Estimate Std. Error t value Pr(>|t|)    
BMI          -2.9845     0.1846  -16.17   <2e-16 ***
\end{verbatim}
\end{exampleblock}
\end{frame}

\begin{frame}
  \begin{center}
    \Large
    Vamos agora fazer predições sobre valores não observados
  \end{center}
\end{frame}

\begin{frame}{E o BMI = 28?}
  \begin{center}
    \includegraphics[height=.9\textheight]{Cap18-19/pratica-plot3}
  \end{center}
\end{frame}

\begin{frame}{E o BMI = 28?}
  \begin{columns}
    \begin{column}{5cm}
      \begin{itemize}
        \small
      \item o valor predito pelo modelo é 15.25169
      \item P: O que isto significa?
      \end{itemize}
    \end{column}
    \begin{column}{5cm}
      \begin{center}
        \includegraphics[width=1.1\textwidth]{Cap18-19/pratica-plot3}
      \end{center}
    \end{column}
  \end{columns}
\end{frame}

\begin{frame}
  \begin{center}
    \Large
    E quando os resíduos não são aleatórios em torno da reta?
  \end{center}
  \vfill
  \hfill \footnotesize (casos extremos)
\end{frame}

\begin{frame}{\small Dispersão em torno da reta {\bf cresce} ao longo da faixa (R2 = 42\%)}
  \begin{center}
    \includegraphics[height=.9\textheight]{Cap18-19/pratica-plot-heterocedasticidade}
  \end{center}
\end{frame}

\begin{frame}{\small Dispersão em torno da reta {\bf varia} ao longo da faixa (R2 = 29\%)}
  \begin{center}
    \includegraphics[height=.9\textheight]{Cap18-19/pratica-plot-heterocedasticidade-sin}
  \end{center}
\end{frame}

\begin{frame}{\small Análise de resíduos dos 2 últimos exemplos}
  \begin{center}
    \includegraphics[height=.6\textheight]{Cap18-19/pratica-plot-heterocedasticidade-resid}
    \includegraphics[height=.6\textheight]{Cap18-19/pratica-plot-heterocedasticidade-sin-resid}

    \vfill
    \scriptsize
    Lembre-se: Ao ler um artigo, você não terá acesso a estas visualizações!
  \end{center}
\end{frame}

\begin{frame}{\small Regressão linear simples x múltipla}
  \begin{block}{Outros fatores}
    Nesses casos, não podemos explicar a variância do BMD apenas com o BMI.

    \bigskip
    \small
    É evidente que algum outro fator {\it deveria} ter sido considerado no modelo

    \bigskip
    Isto permite \alert{ajustar} a heterogeneidade da variância observada com outros cofatores (além do BMI).
  \end{block}
  \vfill
  \hfill \footnotesize (sai a regressão linear {\bf simples} e entra a {\bf múltipla})
\end{frame}

\subsection{Bônus: preditor categórico}

\begin{frame}{\small Exercício da aula de teste t}
  \begin{exampleblock}{}
    \small
    Queremos avaliar a eficiência de uma nova dieta reduzida em
    gordura no tratamento de obesidade.

    \bigskip
    {\footnotesize
      Selecionamos aleatoriamente 100 pessoas obesas para o grupo 1, que receberão a dieta com pouca gordura.
      Selecionamos outras 100 pessoas obesas para o grupo 2 que receberão a mesma quantidade de comida, com proporção normal de gordura.
      O estudo durou 4 meses.
    }

    \bigskip
    \begin{exampleblock}{}
      \footnotesize
      A perda de peso média no grupo 1 foi de 9.33 lbs
      (s=4.72) e no grupo 2 foi de 7.58 lbs (s=3.90).
    \end{exampleblock}
  \end{exampleblock}
  \begin{block}{}
    Essa nova dieta é eficaz na perda de peso?
  \end{block}
  \hfill {\footnotesize Fonte: Khan Academy}
\end{frame}

\begin{frame}[fragile]{\small Resolução com Regressão linear simples}
  \begin{exampleblock}{Saída típica de um programa}
    \tiny
\begin{verbatim}
Residuals:
     Min       1Q   Median       3Q      Max 
-13.7754  -3.1275  -0.2171   3.0112  11.9957 

Coefficients:
            Estimate Std. Error t value Pr(>|t|)    
(Intercept)   9.3340     0.4332  21.548  < 2e-16 ***
GrupoGrupo2  -1.7587     0.6126  -2.871  0.00454 ** 
---
Signif. codes:  0 ‘***’ 0.001 ‘**’ 0.01 ‘*’ 0.05 ‘.’ 0.1 ‘ ’ 1

Residual standard error: 4.332 on 198 degrees of freedom
Multiple R-squared:  0.03996,	Adjusted R-squared:  0.03512 
F-statistic: 8.242 on 1 and 198 DF,  p-value: 0.004537
\end{verbatim}
  \end{exampleblock}
  \begin{exampleblock}{\small Interpretação (assumindo pareamento)}
    \begin{itemize}
      \scriptsize
    \item Perda média do grupo 1 (referência): 9.33 lbs {\tiny (IC=[8.48, 10.19])}.
    \item Perda média do grupo 2 em relação à referência: -1.76 lbs {\tiny (IC=[-2.97, -0.55])}.
    \end{itemize}
  \end{exampleblock}
\end{frame}

\subsection{Resumo}

\begin{frame}{Resumo}
  \begin{itemize}
  \item Quão bem a reta regressora se ajusta aos dados?
    \medskip
  \item O que pode explicar a relação observada?
    \medskip
  \item Qual proporção (porcentagem) da variabilidade pode ser
    explicada pelas variáveis analisadas?
    \medskip
  \item É necessário investigar a relação entre as variáveis!
    \bigskip
    \bigskip
  \item O modelo de RLS permite preditor categórico

    {\footnotesize (com qualquer número de níveis!)}
  \end{itemize}
\end{frame}

\section{Aprofundamento}

\subsection{Aprofundamento}

\begin{frame}{Aprofundamento}
  \begin{block}{Leitura obrigatória}
    \begin{itemize}
      \footnotesize
    \item Capítulo 18
    \item Capítulo 19, pular as seções:
      \begin{itemize}
        \scriptsize
      \item regressão linear como método de mínimos quadrados
      \item calculando a regressão linear
      \end{itemize}
    \end{itemize}
  \end{block}
  \begin{block}{Exercícios selecionados}
    \footnotesize
    Capítulo 19, problemas: todos menos o problema 5.
  \end{block}
  \begin{block}{Leitura recomendada}
    \small
    
    \begin{itemize}
    \item {\bf Capítulo 31}.
      \scriptsize
    \item Schneider A, Hommel G, Blettner M, 2010.

      \url{https://www.ncbi.nlm.nih.gov/pmc/articles/PMC2992018/}
    % \item (paper do exercício) Asomaning, et al., 2006.

    %   \url{https://www.ncbi.nlm.nih.gov/pubmed/17125421}
    \end{itemize}
  \end{block}
\end{frame}

\end{document}
